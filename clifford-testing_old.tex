\documentclass[11pt]{article}

\usepackage{fullpage}
\usepackage{amsmath,amsfonts,amsthm,mathrsfs,mathpazo,xspace,hyperref,graphicx}
\usepackage{endnotes}
\usepackage{color}
\usepackage{bm}
\usepackage{times}
\usepackage{amssymb,latexsym}
\usepackage{float}

\newtheorem{theorem}{Theorem}
\newtheorem{proposition}[theorem]{Proposition}
\newtheorem{conjecture}[theorem]{Conjecture}
\newtheorem{lemma}[theorem]{Lemma}
\newtheorem{claim}[theorem]{Claim}
\newtheorem{fact}[theorem]{Fact}
\newtheorem{corollary}[theorem]{Corollary}

\theoremstyle{remark}
\newtheorem{remark}[theorem]{Remark}

\theoremstyle{definition}
\newtheorem{definition}[theorem]{Definition}
\newtheorem{example}[theorem]{Example}

\newcommand{\beq}{\begin{eqnarray}}
\newcommand{\eeq}{\end{eqnarray}}

\newcommand{\ket}[1]{|#1\rangle}
\newcommand{\bra}[1]{\langle#1|}
\newcommand{\proj}[1]{\ket{#1}\!\bra{#1}}
\newcommand{\Tr}{\mbox{\rm Tr}}
\newcommand{\Id}{\ensuremath{\mathop{\rm Id}\nolimits}}
\newcommand{\Es}[1]{\ensuremath{\mathop{\textsc{E}}_{#1}}}

\newcommand{\CON}{C}
\newcommand{\DIS}{d}
\newcommand{\Drho}{\DIS_\rho}
\newcommand{\Trho}{\Tr_\rho}

\newcommand{\reg}[1]{{\textsf{#1}}}

\newcommand{\C}{\ensuremath{\mathbb{C}}}
\newcommand{\N}{\ensuremath{\mathbb{N}}}
\newcommand{\F}{\ensuremath{\mathbb{F}}}
\newcommand{\K}{\ensuremath{\mathbb{K}}}
\newcommand{\R}{\ensuremath{\mathbb{R}}}
\newcommand{\Z}{\ensuremath{\mathbb{Z}}}

\newcommand{\mH}{\mathcal{H}}
\newcommand{\Alg}{\mathcal{A}}

\newcommand{\setft}[1]{\mathrm{#1}}
\newcommand{\Density}{\setft{D}}
\newcommand{\Pos}{\setft{Pos}}
\newcommand{\Proj}{\setft{Proj}}
\newcommand{\Obs}{\setft{Obs}}
\newcommand{\Channel}{\setft{C}}
\newcommand{\Unitary}{\setft{U}}
\newcommand{\Herm}{\setft{Herm}}
\newcommand{\Lin}{\setft{L}}
\newcommand{\Trans}{\setft{T}}
\DeclareMathOperator{\poly}{poly}

\newcommand{\eps}{\varepsilon}
\newcommand{\ph}{\ensuremath{\varphi}}

\newcommand{\Acc}{\textsc{Acc}}
\newcommand{\Samp}{\textsc{Samp}}
\newcommand{\Ext}{\ensuremath{\text{Ext}}}

\newcommand{\Hmin}{H_\infty}
\newcommand{\Hmax}{H_{\ensuremath{\text{max}}}}

\newcommand{\CHSH}{{\rm CHSH}}
\newcommand{\EPR}{{\rm EPR}}
\newcommand{\MS}{{\rm MS}}
\newcommand{\basis}{\mathcal{B}}
\newcommand{\pauli}{\mathcal{P}}
\newcommand{\paulip}{\tilde{\mathcal{P}}}
\newcommand{\pbt}{\textsc{pbt}}
\newcommand{\qauli}{\mathcal{Q}}
\newcommand{\rauli}{\mathcal{R}}
\newcommand{\raulip}{\tilde{\mathcal{R}}}
\newcommand{\qaulip}{\tilde{\mathcal{Q}}}
\newcommand{\magic}{\mathcal{M}}
\newcommand{\wagon}{\mathcal{W}}
\newcommand{\aux}{\textsc {aux}}
\newcommand{\ctl}{\textsc {ctl}}
\newcommand{\swap}{\textsc {swap}}
\newcommand{\conj}{\textsc{conj}}
\newcommand{\perm}{\textsc{tens}}
\newcommand{\prodt}{\textsc{prod}}
\newcommand{\comt}{\textsc{com}}
\newcommand{\act}{\textsc{ac}}
\newcommand{\idt}{\textsc{id}}
\newcommand{\bellt}{\textsc{Bell}}

\newcommand{\conjc}{\textsc{conj-cliff}}
\newcommand{\ecliff}{\textsc{e-cliff}}
\newcommand{\cliff}{\textsc{cliff}}
\newcommand{\SWAP}{\textsc{SW}}
\newcommand{\clifford}{\mathcal{C}}
\newcommand{\cliffordga}{{\mathcal{C}_1}}
\newcommand{\cliffordgb}{{\mathcal{C}_2}}
\newcommand{\heisg}{{\mathcal{H}^{(1)}}}
\newcommand{\heisgn}{{\mathcal{H}^{(n)}}}
\newcommand{\heisga}{{\mathcal{H}_1}}
\newcommand{\heisgb}{{\mathcal{H}_{2}}}
\newcommand{\cliffordgan}{{\mathcal{C}^{(n)}_1}}
\newcommand{\cliffordgbn}{{\mathcal{C}^{(n)}_2}}
\newcommand{\cliffordn}{G_\mathcal{C}^{(n)}}
\newcommand{\paulig}{G_{\mathcal{P}}}
\newcommand{\pauligb}{G_{\mathcal{P}_2}}
\newcommand{\paulign}{G_{\mathcal{P}}^{(n)}}
\newcommand{\conjn}{\mathcal{J}^{(n)}\!}
\newcommand{\conjr}{\mathcal{J}\!}
\newcommand{\tr}{\mathcal{T}\!}
\newcommand{\epaulin}{\hat{\mathcal{P}}^{(n)}\!}
\newcommand{\tpaulin}{\tilde{\mathcal{P}}^{(n)}\!}
\newcommand{\paulin}{\mathcal{P}^{(n)}\!}
\newcommand{\ver}{\textsc{V}}
\newcommand{\pv}{\textsc{PV}}
\newcommand{\pp}{\textsc{PP}}
\newcommand{\sk}{\ensuremath{\text{sk}}}

\newcommand{\anote}[1]{\textcolor{blue}{\small {\textbf{(Andrea:} #1 \textbf{) }}}}
\newcommand{\agnote}[1]{\textcolor{cyan}{\small {\textbf{(Alex:} #1 \textbf{) }}}}
\newcommand{\agnew}[1]{\textcolor{cyan}{#1}}


\bibliographystyle{alpha}

\newif\ifnotes\notestrue
%\newif\ifnotes\notesfalse

%\input{../marginnotes}
\input{marginnotes}

\begin{document}

\title{Robust testing of Clifford observables}
\author{}
\date{}
\maketitle

\noteswarning

%=====================%
\section{Introduction}
\label{sec:intro}
%=====================%

\tnote{Generic testing will give it, but group has exponential size, so soundness could be exponentially small}\tnote{Not clear if just Pauli conjugation is enough; also relation between Cliffords?}

%=====================%
\section{Preliminaries}
\label{sec:prelim}
%=====================%


\subsection{Notation}
\label{sec:prelim-notation}

$\mH$ will always denote a finite-dimensional Hilbert space.  We write $\Unitary(\mH)$ for the set of unitary operators, $\Obs(\mH)$ for the set of observables and $\Proj(\mH)$ the set of projective measurements on $\mH$ respectively.  


\subsection{Pauli and Clifford groups}
\label{sec:groups}

We use capital letters $X,Z,W,\ldots$ to denote observables. We use greek letters $\sigma$, $\tau$ with a subscript $\sigma_W$, $\tau_W$, to emphasize that the observable $W$ specified as subscript acts in a particular basis. For example, $X$ is an arbitrary observable but $\sigma_X$ is specifically the Pauli $X$ matrix defined in~\eqref{eq:pauli-matrix}. 

For $a\in\{0,1\}^n$ and commuting observables $\sigma_{W_1},\ldots,\sigma_{W_n}$, we write $\sigma_W(a) = \prod_{i=1}^n (\sigma_{W_i})^{a_i}$. The associated projective measurements are $\sigma_{W_i} = \sigma_{W_i}^0 - \sigma_{W_i}^1$ and $\sigma_W^u = \Es{a} (-1)^{u\cdot a} \sigma_W(a)$.  Often the $\sigma_{W_i}$ will be single-qubit observables acting on distinct qubits, in which case each is implicitly tensored with identity outside of the qubit on which it acts. 
We sometimes consider qubits in pairs, and may write  $\sigma_W(a,a')$ for $a,a'\in\{0,1\}^n$. Which string indexes which qubits should always be clear from context. 

Let 
\begin{equation}\label{eq:pauli-matrix}
\sigma_I = \begin{pmatrix} 1 & 0 \\ 0 & 1 \end{pmatrix},\qquad \sigma_X = \begin{pmatrix} 0 & 1 \\ 1 & 0 \end{pmatrix},\qquad \sigma_Y = \begin{pmatrix} 0 & i \\ -i & 0 \end{pmatrix}\qquad\text{and}\qquad \sigma_Z = \begin{pmatrix} 1 & 0 \\ 0 & -1\end{pmatrix}
\end{equation}
denote the standard Pauli matrices acting on a qubit. The single-qubit Weyl-Heisenberg group
$$\heisg = H(\Z_2)=\Big\{(-1)^c\sigma_X(a)\sigma_Z(b),\;a,b,c\in\{0,1\}\Big\} $$
is the Pauli group quotiented by the two-element subgroup generated by the complex phase $i$.\footnote{See e.g.~\cite[Section II.A]{gross2006hudson}.} We  let $\heisgn = H(\Z_2^n)$ be the direct product of $n$ copies of $\heisg$.  
%We use $\paulign = \{ \pm\Id,\pm \sigma_X,\pm \sigma_Z,\pm\sigma_Y\}^{\otimes n}$ to denote the $n$-qubit Pauli group. 
The $n$-qubit Clifford group is the normalizer of $\heisgn$ in the unitary group, up to phase: 
$$\cliffordn = \big\{G\in\Unitary((\C^2)^{\otimes n}):\, G \sigma G^\dagger \in \{\pm 1,\pm i\}\heisgn \quad\forall \sigma \in \heisgn\big\}.$$
%An element $H$ of the Clifford group is uniquely characterized by functions $h_S:\{0,1\}^n\times \{0,1\}^n \to \Z_4$ and $h_X,h_Z:\{0,1\}^n \times \{0,1\}^n \to \{0,1\}^n$ such that $H \sigma_X(a)\sigma_Z(b) H^\dagger = i^{h_S(a,b)} \sigma_X(h_X(a,b))\sigma_Z(h_Z(a,b))$ for all $(a,b)\in\{0,1\}^n\times \{0,1\}^n$. 
%The Clifford group has $24$ elements. Each element can be characterized by a $2\times 2$ symplectic matrix over $\Z_4$ which describes its action on the Pauli group. 
Some Clifford observables we will use include 
$$ \sigma_H = \frac{\sigma_X+\sigma_Z}{\sqrt{2}},\qquad \sigma_{H'} = \frac{\sigma_X-\sigma_Z}{\sqrt{2}},\qquad \sigma_F = \frac{-\sigma_X+\sigma_Y}{\sqrt{2}},\qquad \sigma_{G} = \frac{\sigma_X+\sigma_Y}{\sqrt{2}}.$$
Note that  $\sigma_H$ and $\sigma_{H'}$ are characterized by $\sigma_X \sigma_H \sigma_X = \sigma_{H'}$ and $\sigma_Z \sigma_H \sigma_Z = -\sigma_{H'}$. Similarly, $\sigma_F$ and $\sigma_G$ are characterized by $\sigma_X \sigma_F \sigma_X = -\sigma_G$ and $\sigma_Y \sigma_F \sigma_Y = \sigma_G$. 

%============================%
\subsection{Testing}
\label{sec:general-rigidity}
%============================%

\paragraph{Distance measures.}
Ultimately our goal is to test that a prover implements a certain tensor product of single-qubit or two-qubit measurements defined by observables such as $\sigma_X$, $\sigma_Y$, or $\sigma_G$. Since it is impossible to detect whether a prover applies a certain operation $X$ on state $\ket{\psi}$, or $VXV^\dagger$ on state $V\ket{\psi}$, for any isometry $V:\Lin(\mH)\to\Lin(\mH')$ such that $V^\dagger V = \Id$, we will (as is standard in testing) focus on testing identity up to \emph{local isometries}. Statements of ``approximate equivalence up to local isometry'' can be tedious to formulate, and we introduce the following important piece of notation: 

\begin{definition}
For finite-dimensional Hilbert spaces $\mH$ and $\mH'$, $\delta>0$, and operators $R \in\Lin(\mH)$ and $S\in\Lin(\mH')$ we say that $R$ and $S$ are $\delta$-isometric with respect to $\ket{\psi} \in \mH \otimes \mH$, and write $R\simeq_\delta S$, if there exists an isometry $V:\mH\to\mH'$ such that 
$$\big\|( R-V^\dagger SV)\otimes \Id \ket{\psi}\big\|=O(\delta).$$
If $V$ is the identity then we further say that $R$ and $S$ are $\delta$-equivalent, and write $R\approx_\delta S$ for $\| ( R- S) \otimes \Id \ket{\psi}\|=O(\delta)$.
\end{definition}

The notation $R\simeq_\delta S$ carries some ambiguity, as it does not specify the state $\ket{\psi}$. The latter should always be clear from context: we will often simply write that $R$ and $S$ are $\delta$-isometric, without explicitly specifying $\ket{\psi}$ or the isometry. The relation is transitive, but not reflexive: the operator on the right will always act on a space of dimension at least as large as that on which the operator on the left acts. The notion of $\delta$-equivalence is both transitive (it obeys the triangle inequality) and reflexive, and we will use it as our main notion of distance. 

\paragraph{Tests.}
We formulate our tests as two-player games in which both players are treated symmetrically.  We often use the same symbol, a capital letter $X,Z,W,\ldots,$ to denote a question in the game and the associated projective measurement $\{W^a\}$ applied by the player upon receipt of that question. To a projective measurement with outcomes in $\{0,1\}^n$ we  associate a family of observables $W(u)$ parametrized by $n$-bit strings $u\in\{0,1\}^n$ and defined as $W(u) = \sum_a (-1)^{u\cdot a} W^a$. If $n=1$ we simply write $W=W(1)=W^0-W^1$; note that $W(0)=\Id$.

The games we consider always implicitly include a ``consistency test'', played with probability $1/2$, in which the verifier sends the same random question to each player (chosen according to a distribution which matches the marginal distribution from the remaining, ``main'' part of the test) and accepts their answers if and only if they are identical. Success with probability at least $1-\eps$ in this test, for a projective measurement $\{W^a\}$, implies 
\begin{align*}
\sum_a \|(W^a \otimes \Id - \Id \otimes W^a)\ket{\psi}_{\reg{AB}}\|^2 &= 2-2 \sum_a \bra{\psi} W^a \otimes W^a \ket{\psi} \\ 
&\leq 2 \eps,
\end{align*}
so that $W^a \otimes \Id \approx_{\sqrt{\eps}} \Id \otimes W^a$. Similarly, if $W$ is an observable for the players that succeeds in the consistency test with probability $1-\eps$ we obtain $W\otimes \Id \approx_{\sqrt{\eps}} \Id \otimes W$ \agnote{Aren't the two last statements saying the same thing? Or is there a subtlety that I cannot see?}\tnote{They're basically the same except one is formulated in terms of POVM and the other in terms of observables. The first statement is more general, but the second is the one that is most frequently used, so I thought it would be helpful to make it explicit}. We will often use both relations to ``switch'' operators from one player's space to the other's; as a result we will also often omit to  explicitly specify on which player's space an observable is applied. 

\paragraph{Strategies.} Given a two-player game, or test, a strategy for the players is constituted of a bipartite entangled state $\ket{\psi} \in \mH_\reg{A} \otimes \mH_\reg{B}$ together with families of projective  measurements $\{W^a_\reg{A}\}$ for Alice and $\{W_\reg{B}^a\}$ for Bob, one for each question $W$ that can be sent to either player in the test.\footnote{We make the assumption that the provers employ a pure-state strategy for convenience, but it is easy to check that all proofs extend to the case of a mixed strategy. It is always possible to consider projective strategies only by applying Naimark's dilation theorem, adding an auxiliary local system to each player as necessary, since no bound is assumed on the dimension of their systems.}  %The fact that both players will always be treated symmetrically allows us to assume without loss of generality that $\mH_\reg{A} = \mH_\reg{B}$, $\ket{\psi}$ is permutation-invariant, and for any question $W$, $W_\reg{A}^a = W_\reg{B}^a$ for all answers $a$ (see e.g. Lemma~4 in~\cite{kempe2011entangled}).
We will loosely refer to a strategy for the players as $(W,\ket{\psi})$, with the symbol $W$ referring to the complete set of POVM used by the players in the game; taking advantage of  symmetry we often omit the subscript $\reg{A}$ or $\reg{B}$, as all statements involving observables for one player should hold verbatim with the other player's observables as well. 

\paragraph{Relations.}
We use $\mathcal{R}$ to denote a set of relations over variables $X,Z,W,\ldots,$ such as
$$\mathcal{R}=\big\{XZXZ=-\Id,\, HX=ZH,\,X,Z,H\in\Obs\big\}.$$
We only consider relations that can be brought in the form either $W_1\cdots W_k = (-1)^a \Id$, where the $W_i$ are (not necessarily distinct) unitary variables and $a\in\Z_2$, or $W ( \sum_a \omega_a V^a)=\Id$, where $W$ is a unitary variable, $\{V^a\}$ a POVM with $A$ outcomes, and $\omega_a$ are $A$-th roots of unity.   

\begin{definition}[Rigid self-test]
We say that a set of relations $\mathcal{R}$ is $(c,\delta(\eps))$-testable, on the average under distribution $\mathcal{D}:\mathcal{R}\to[0,1]$, if
  there exists a game (or test) $G$ with question set $\mathcal{Q}$ that
  includes (at least) a symbol for each variable in $\mathcal{R}$ that is either an observable or a POVM and such that:
\begin{itemize}
\item (\emph{Completeness}) There exists a set of operators which exactly satisfy all relations in $\mathcal{R}$ and a strategy for the players which uses these operators (together possibly with others for the additional questions) that has success probability at least $c$;
\item (\emph{Soundness}) For any $\eps>0$ and any strategy $(W,\ket{\psi}_{AB})$ that succeeds in the game with probability at least $c-\eps$ the associated measurement operators satisfy the relations in $\mathcal{R}$ up to $\delta(\eps)$, in the state-dependent norm. More precisely, on average
 over the choice of a relation $f(W)=(-1)^a\Id$ from $\mathcal{R}$ chosen according to $\mathcal{D}$, it holds that $\| \Id\otimes (f(W)-(-1)^a\Id) \ket{\psi}_{\reg{AB}}\|^2 \leq \delta(\eps)$.
\end{itemize}
If both conditions hold we also say that the game $G$ is a robust $(c,s)$ self-test for the relations $\mathcal{R}$. 
\end{definition}

\tnote{should discuss connection with BCS}
Even though our definition is general, in this paper we only consider games with perfect completeness, $c=1$, so that we usually omit the parameter. The distribution $\mathcal{D}$ will often be implicit from context, and we do not always specify it explicitly (e.g. in case we only measure $\delta(\eps)$ up to multiplicative factors of order $|\mathcal{R}|$ the exact distribution $\mathcal{D}$ does not matter as long as it has complete support). 

\begin{definition}[Stable relations]
We say that a set of relations $\mathcal{R}$ is $\delta(\eps)$-stable, on the average under distribution $\mathcal{D}:\mathcal{R}\to[0,1]$, if for any operators $W\in\Lin(\mH)$ that satisfy the relations, on average, up to error at most $\eps$ in state-dependent norm (for any fixed state), i.e. 
$$\Es{\substack{(f,a)\sim\mathcal{D}:\\f(W)=(-1)^a\Id \in\mathcal{R}}} \big\| \Id\otimes (f(W)-(-1)^a \Id) \ket{\psi}\big\|^2 \leq \eps,$$
and moreover the $W$ are self-consistent, i.e.
$$ \Es{W} \big\| (\Id\otimes W - W \otimes \Id)\ket{\psi}\big\|^2\leq \eps,$$
  there exists operators $\hat{W}$ which satisfy the same relations exactly and are $\delta(\eps)$-isometric to the $W$ with respect to $\ket{\psi}$, on the average over the choice of a random relation in $\mathcal{R}$ and a uniformly random $W$ appearing in the relation, i.e. there exists an isometry $V$ such that 
  \begin{equation}
    \Es{R \sim\mathcal{D}} \Es{W \in_U R} \big\|( \hat{W}-V^\dagger WV)\otimes \Id \ket{\psi}\big\|=O(\delta(\eps)), \nonumber
  \end{equation}
where $W \in R$ is shorthand for $W$ appearing in relation $R$.
	\end{definition}

\subsection{The Magic Square game}
\label{sec:ms}

We use the Magic Square game~\cite{Mermin90} as a building block, noting that it  provides a robust self-test test for the two-qubit Weyl-Heisenberg group (see Section~\ref{sec:groups} for the definition). Questions in this game are specified by a triple of labels corresponding to the same row or column from the square pictured in Figure~\ref{fig:ms} (so a typical question could be $(IZ,XI,XZ)$; there are $6$ questions in total, each a triple). An answer is composed of three values in $\{\pm 1\}$, one for each of the labels making up the question. Answers from the player should be entrywise consistent, and such that the product of the answers associated to any row or column except the last should be $+1$; for the last column it should be $-1$. The labels indicate the ``honest'' strategy for the game, which consists of each player measuring two half-EPR pairs using the commuting Pauli observables indicated by the labels of his question. 

\begin{figure}[H]
\begin{center}
\begin{tabular}{|c|c|c|}
\hline
$IZ$ & $ZI$ & $ZZ$ \\
\hline
$XI$ & $IX$ & $XX$ \\
\hline
$XZ$ & $ZX$ & $YY$\\
\hline
\end{tabular}
\end{center}
\caption{Questions, and a strategy, for the Magic Square game}
\label{fig:ms}
\end{figure}

The following lemma states some consequences of the Magic Square game, interpreted as a self-test (see e.g.~\cite{WBMS16}). 

\begin{lemma}\label{lem:ms-rigid}
Suppose a strategy for the players succeeds with probability at least $1-\eps$ in the Magic Square game. Then there exist  isometries $V_D:\mH_\reg{D} \to (\C^2\otimes \C^2)_{\reg{D'}}\otimes \hat{\mH}_{\hat{\reg{D}}}$, for $D\in\{A,B\}$, such that
$$\big\| (V_A \otimes V_B) \ket{\psi}_{\reg{AB}} - \ket{\EPR}_{\reg{A}'\reg{B}'}^{\otimes 2} \ket{\aux}_{\hat{\reg{A}}\hat{\reg{B}}} \big\|^2 = O(\sqrt{\eps}),$$
and for $W\in \{I,X,Z\}^2 \cup \{YY\}$,
\begin{align*}
\big\| \big(W -V_A^\dagger \sigma_W V_A\big) \otimes \Id_B \ket{\psi} \big\|^2 &= O(\sqrt{\eps}).
\end{align*}
\end{lemma}

%===========================%
\section{Elementary tests}
\label{sec:clifford-test}
%===========================%

In this section, we collect simple tests that will be used as building blocks. We start in Section~\ref{sec:elementary} by reviewing elementary tests whose analysis is either immediate or can be found in the literature. In Section~\ref{sec:bell} we formulate a simple test for measurements in the Bell basis and the associated two-qubit SWAP observable. In Section~\ref{sec:conj-test} we give a test for conjugation of an observable by a unitary. 

%---------------------------%
\subsection{Elementary tests}
\label{sec:elementary}
%---------------------------%


Figure~\ref{fig:elementary} summarizes some useful elementary tests. For each test, ``Inputs'' refers to a subset of designated questions in the test; ``Relation'' indicates a relation that the test aims to certify; ``Test'' describes the certification protocol. (Recall that all our protocols implicitly include a ``consistency'' test in which a question is chosen uniformly at random from the marginal distribution and sent to both players, whose answers are accepted if and only if they are equal.)

\begin{figure}[H]
\rule[1ex]{16.5cm}{0.5pt}\\
Test~$\idt(A,B)$:
\begin{itemize}
    \item Inputs: $A$, $B$ two observables on the same space $\mH$
    \item Relation: $A=B$
    \item Test: Send $W \in \{A,B\}$ and $W'\in\{A,B\}$, chosen uniformly at random, to the first and second player respectively. Receive an answer in $\{\pm 1\}$ from each player. Accept if and only if the answers are equal whenever the questions are identical. \agnote{Shouldn't we always check the answers since we want to say that A = B?}\tnote{Right, this is what the test is doing, checking the answers are the same when the questions are the same. But sometimes the questions are different. There is no real motivation to do this though. Maybe we should get rid of the test anyways, I put it a an illustration, but it's never used}
\end{itemize}
Test~$\act(X,Z)$:
\begin{itemize}
    \item Inputs: $X$, $Z$ two observables on the same space $\mH$
    \item Relation: $XZ=-ZX$
    \item Test: Execute the Magic Square game, using the label ``$X$'' for the $(2,1)$ entry and ``$Z$'' for the $(1,2)$ entry.  
\end{itemize}
Test~$\comt(A,B)$:
\begin{itemize}
    \item Inputs: $A$, $B$ two observables on the same space $\mH$.
    \item Relation: $AB=BA$
    \item Test: Send $W\in\{A,B\}$ chosen uniformly at random to the first player. Send $(A,B)$ to the second player. Receive a bit $c\in\{\pm 1\}$ from the first player, and two bits $(a',b')\in\{\pm 1\}^2$ from the second. Accept if and only if $c=a'$ if $W=A$, and $c=b'$ if $W=B$. 
\end{itemize}
Test~$\prodt(A,B,C)$:
\begin{itemize}
    \item Inputs: $A$, $B$ and $C$ three observables on the same space $\mH$.
    \item Relations: $AB=BA=C$
    \item Test: Similar to the commutation game, but use $C$ to label the question $(A,B)$.
\end{itemize}
\rule[2ex]{16.5cm}{0.5pt}\vspace{-1cm}
\caption{Some elementary tests.}
\label{fig:elementary}
\end{figure}

\begin{lemma}\label{lem:elementary}
Each of the tests described in Figure~\ref{fig:elementary} is a robust $(1,\delta)$ self-test for the indicated relation(s), for some $\delta = O(\eps^{1/2})$. 
\end{lemma}

\begin{proof}
The proof for each test is similar. As an example we give it for the commutation test $\comt(A,B)$. 

First we verify completeness. Let $A,B$ be two commuting observables on $\mH$, and $\ket{\EPR}_{\reg{AB}}$ the maximally entangled state in $\mH_\reg{A}\otimes \mH_\reg{B}$. Upon receiving question $A$ or $B$, the player measures the corresponding observable. If the question is $(A,B)$, he jointly measures $A$ and $B$. This strategy succeeds with probability $1$ in the test. 

Next we establish soundness. Let $\ket{\psi} \in \mH_{\reg{A}}\otimes \mH_\reg{B}$ be a state shared by the provers, $A$, $B$ their observables on questions $A,B$, and $\{C^{a,b}\}$ the four-outcome POVM applied on question $(A,B)$. Assume the strategy succeeds with probability at least $1-\eps$. Recall that this includes both the test described in Figure~\ref{fig:elementary}, and the automatic consistency test; recall also that Naimark's dilation theorem allows us to assume all POVM used by the players are projective. Let $C_A = \sum_{a,b} (-1)^a C^{a,b}$ and $C_B = \sum_{a,b} (-1)^b C^{a,b}$. Then $C_A$ and $C_B$ commute. Thus
\begin{align*}
A_\reg{A} B_\reg{A} \otimes \Id_\reg{B}
&\approx_{\sqrt{\eps}} A_\reg{A} \otimes (C_B)_\reg{B}\\
&\approx_{\sqrt{\eps}} \Id_\reg{A} \otimes (C_B)_\reg{B}(C_A)_\reg{B}\\
&=  \Id_\reg{A} \otimes (C_A)_\reg{B}(C_B)_\reg{B}\\
&\approx_{\sqrt{\eps}} B_\reg{A} \otimes (C_A)_\reg{B}\\
&\approx_{\sqrt{\eps}} B_\reg{A} A_\reg{A} \otimes \Id_\reg{B}.
\end{align*}
Here each approximation uses the consistency condition provided by the test. This shows that $A$ and $B$ approximately satisfy the commutation relation, in the state-dependent distance. 
\end{proof}



%---------------------------%
\subsection{The Bell basis}
\label{sec:bell}
%---------------------------%

Given two commuting pairs of anti-commuting observables $\{X_1,Z_1\}$ and $\{X_2,Z_2\}$ we provide a test for a four-outcome projective measurement in the Bell basis (i.e. the joint eigenbasis of $X_1X_2$ and $Z_1Z_2$). The same test can be easily extended to test for the $\SWAP$ observable, which exchanges two qudits. The Bell measurement test described in Figure~\ref{fig:bell} tests for both. 


\begin{figure}[H]
\rule[1ex]{16.5cm}{0.5pt}\\
Test~$\bellt(X_1,X_2,Z_1,Z_2)$:
\begin{itemize}
    \item Inputs: For $i\in\{1,2\}$, $\{X_i,Z_i\}$ observables, $\{\Phi^{ab}\}_{a,b\in\{0,1\}}$ a four-outcome projective measurement, and $\SWAP$ an observable, all acting on the same space $\mH$.
    \item Relations: for all $a,b\in\{0,1\}$, $\Phi^{ab} = \frac{1}{4}\big(\Id+(-1)^a Z_1Z_2\big)\big(\Id+(-1)^bX_1X_2\big)$, and $\SWAP = \Phi^{00} + \Phi^{01} + \Phi^{10} - \Phi^{11}$.   \agnote{Is it possible to rewrite the relations in the form $W_1...W_k =I$?}\tnote{You're right, thanks. I thought so but due to the $\SWAP$ having non-zero trace this is not so obvious. I changed the definition of relations to accommodate...Maybe this definition is doing more harm than good, since in the end many of the tests don't directly test relations. What I wanted is a succinct vocabulary that avoided the use of ``isometry'' in every single lemma...}
    \item Test: execute each of the following with equal probability:
		\begin{enumerate}
		\item[(a)] Execute the Magic Square game, labeling each entry of the square from Figure~\ref{fig:ms} (except entry $(3,3)$) using the observables $X_1,Z_1$ and $X_2,Z_2$.
		\item[(b)] Send $\Phi$ \agnote{Is $\Phi$ the observable or $\Phi^{ab}$?}\tnote{$\Phi$ is the question, with the intention being that the prover measures the four-outcome POVM associated to it} to one prover and the labels $(X_1X_2,Z_1Z_2,Y_1Y_2)$ associated with the third column of the Magic Square to the other.  The first player replies with $a,b\in\{0,1\}$, and the second with $c,d,e\in \{\pm 1\}$. The referee checks the players' answers for the obvious consistency conditions. For example, if the first player reports the outcome $(0,0)$, then the referee rejects if $(c,d)\neq (+1,+1)$. 
		\item[(c)] Send $\Phi$ to one prover and $\SWAP$ to the other. The first player replies with $a,b\in\{0,1\}$, and the second with $c\in \{\pm 1\}$. Accept if and only $c=(-1)^{ab}$. 
		\end{enumerate}
\end{itemize}
\rule[2ex]{16.5cm}{0.5pt}\vspace{-1cm}
\caption{The Bell measurement test.}
\label{fig:bell}
\end{figure}


\begin{lemma}\label{lem:bell-rigid-test}
The test $\bellt(X_1,X_2,Z_1,Z_2)$ is a robust $(1,\delta)$ self-test for 
\begin{align*}
\mathcal{R}&= \Big\{\big\{\Phi^{ab}\big\}_{a,b\in\{0,1\}}\in\Proj,\, \SWAP\in\Obs\Big\}\cup \Big\{\Phi^{ab} = \frac{1}{4}\big(1+(-1)^a Z_1Z_2\big)\big(1+(-1)^bX_1X_2\big) \Big\}\\
&\qquad\qquad \cup \big\{\SWAP = \Phi^{00}+\Phi^{01}+\Phi^{10}-\Phi^{11}\big\},
\end{align*}
 for some $\delta(\eps) = O(\sqrt{\eps})$.
\end{lemma}

%Note that technically the relations $\mathcal{R}$ can be put in the form $W_1\cdots W_k = \Id$ %by introducing observables 
%$$XZ' = \sum_{a,b} e^{\frac{2i\pi(a+2b)}{4}}\frac{1}{4}(\Id + (-1)^a X_1X_2)(\Id + (-1)^b %Z_1Z_2),\qquad \Phi' = \sum_{a,b} e^{\frac{2i(a+2b)\pi}{4}}\Phi^{a,b},$$
%in which case they are equivalent to $XZ' \Phi' = \Id$ and . 

%are also $O(\eps)$-stable: this is an immediate consequence of stability for $\mathcal{P}^{(2)}$ (Lemma~\ref{lem:pauli-stable}).

\begin{proof}
Completeness is clear: the players can play the honest strategy for the Magic Square game, use a measurement in the Bell basis on their two qubits for $\Phi$, and measure the SWAP observable for $\SWAP$. 

For soundness, let $\ket{\psi}\in\mH_\reg{A}\otimes \mH_\reg{B}$, $\{W_1W'_2:\, W,W'\in\{I,X,Z\}\}$, $\{\Phi^{ab}\}$ and $\SWAP$ denote a state and operators for a strategy that succeeds with probability at least $1-\eps$ in the test. From the analysis of the Magic Square game (Lemma~\ref{lem:ms-rigid}) it follows that the players' observables $X_1X_2$ and $Z_1Z_2$ associated to questions with those  labels approximately commute, and are each the product of two commuting observables $X_1I$, $IX_2$ and $Z_1I$, $IZ_2$ respectively, such that $X_1I$ and $Z_1I$, and $IX_2$ and $IZ_2$, anti-commute. 

Since $X_1X_2$ and $Z_1Z_2$ appear together in the same question to the player (the last column of the Magic Square~\ref{fig:ms}), the player has a four-outcome projective measurement $\{W^{c,d}\}_{c,d\in\{0,1\}}$ such that $\sum_d (-1)^c W^{c,d} = X_1X_2$ and $\sum_c (-1)^dW^{c,d} = Z_1Z_2$, from which it follows that $W^{c,d} = (1/4)(1+(-1)^c Z_1Z_2)(1+(-1)^d X_1X_2)$. 

The player's success probability in part (b) of the test is then
\begin{align*}
\sum_{a,b} \bra{\psi} \Phi^{ab} \otimes W^{a,b} \ket{\psi} &= \sum_{a,b} \bra{\psi} \Phi^{ab} \otimes \frac{1}{4} \big(1+(-1)^a Z_1Z_2\big)\big(1+(-1)^bX_1X_2\big) \ket{\psi}\\
&\approx_{\sqrt{\eps}} \sum_{a,b} \bra{\psi} \Id \otimes \frac{1}{4} \big(1+(-1)^a Z_1Z_2\big)\big(1+(-1)^bX_1X_2\big)\,\Phi^{a,b} \ket{\psi},
\end{align*}
where the second line uses the implicit consistency test to switch $\{\Phi^{a,b}\}$ from one player's space to the other. The condition that the last expression be at least $1-O(\sqrt{\eps})$ implies 
$\Phi^{ab} \approx_{\sqrt{\eps}}(1/4)(1+(-1)^a Z_1Z_2)(1+(-1)^bX_1X_2)$, as required. 

Finally, part (c) checks for the correct relationship between $\SWAP$ and $\Phi$; the analysis is similar.  
\end{proof}


%---------------------------%
\subsection{The conjugation test}
\label{sec:conj-test}
%---------------------------%

We give a test which certifies that a unitary (not necessarily an observable) conjugates an observable to another. More precisely, let $A,B$ be observables, and $R$ a unitary, acting on the same space $\mH$. The test $\conj(R,C)$ certifies that the players implement observables of the form
\begin{equation}\label{eq:def-xr}
X_R = \begin{pmatrix} 0 & R^\dagger\\ R & 0 \end{pmatrix}\qquad \text{and}\qquad C = C_{A,B} = \begin{pmatrix} A & 0\\ 0 & B \end{pmatrix}
\end{equation}
such that $X_R$ and $C$ commute. The fact that $X_R$ is an observable implies that $R$ is unitary,\footnote{Note that $R$ will not be directly accessed in the test, since by itself it does not necessarily correspond to a measurement.} while the commutation condition is equivalent to the relation $RAR^\dagger = B$. The test thus tests for the relations
\begin{align*}
 \mathcal{C}\{R,C\} &= \big\{ X_R,C,X,Z\in \Obs\big\} \cup \big\{XZ=-ZX\big\}
\cup \big\{ X_R C = C X_R,\, X_RZ=-Z X_R,\, C Z=ZC\big\}.
\end{align*}
Here the anti-commuting observables $X$ and $Z$ are used to specify a basis in which $X_R$ and $C$ can be block-diagonalized. The anti-commutation and commutation relations with $Z$ enforce that $X_R$ and $C$ respectively have the form described in~\eqref{eq:def-xr}: it is an easy exercise to verify that the relations $\mathcal{C}\{R,C\}$ imply that $X_R$ and $C$ can be put in the form~\eqref{eq:def-xr}, with the block structure specified by the $+1$ and $-1$ eigenspaces of $Z$.

\begin{figure}[H]
\rule[1ex]{16.5cm}{0.5pt}\\
Test~\conj(A,B,R) 
\begin{itemize}
    \item Inputs: $A$ and $B$ observables on the same space $\mH$, and $X$ and $Z$ observables on $\mH'$. $X_R$ and $C$ observables on $\mH\otimes \mH'$.
    \item Relation:  $\mathcal{C}\{R,C\} $, with $R$ defined from $X_R$, and $C$ related to $A$ and $B$, as in~\eqref{eq:def-xr}. 
    \item Test: execute each of the following with equal probability
		\begin{enumerate}
\item[(a)] With probability $1/8$ each, execute tests $\act(X,Z)$,  $\comt(C,Z)$, $\comt(X_R,C)$,   $\act(X_R,Z)$ and $\comt(A,X)$, $\comt(B,X)$, $\comt(A,Z)$, $\comt(B,Z)$. 
\item[(b)] Ask one player to measure $A$, $B$, $C$ or $Z$ (with probability $1/4$ each), and the other to jointly measure $A$ or $B$ (with probability $1/2$ each) and $Z$. The first player returns one bit, and the second two bits. Reject if either:
\begin{itemize}
\item The first player was asked $C$, the second player was asked $(A,Z)$, his second answer bit is $0$, and his first answer bit does not match the first players';
\item The first player was asked $C$, the second player was asked $(B,Z)$, his second answer bit is $1$, and his first answer bit does not match the first players'.
\item The first player was asked $A$, $B$, or $Z$ and his answer bit does not match the corresponding answer from the second player.
\end{itemize}
\end{enumerate}
\end{itemize}
\rule[2ex]{16.5cm}{0.5pt}\vspace{-1cm}
\caption{The conjugation test, $\conj(A,B,R)$.}
\label{fig:conjugation-test-1}
\end{figure}

\begin{lemma}\label{lem:conj}
The test $\conj(A,B,R)$ is a $(1,\delta)$ self-test for the set of relations $\mathcal{C}\{R,C\}$, for some $\delta = O(\sqrt{\eps})$. Moreover, $C \approx_{\delta} A (\Id+Z)/2 + B(\Id-Z)/2$. 
\end{lemma}

\begin{proof}
Completeness is clear, as players making measurements on a maximally entangled state on $\mH_{\reg{A}}\otimes \mH_{\reg{B}}$, tensored with an EPR pair on $\C^2 \otimes \C^2$ for the $X$ and $Z$ observables, and using $X_R$ and $C$ defined in~\eqref{eq:def-xr} (with the blocks specified by the EPR pair) succeed in each test with probability $1$. 

We now consider soundness. Success in $\act(X,Z)$ from part (a) of the test implies the existence of local isometries $V_A,V_B$ such that $V_A:\mH_\reg{A}\to \mH_{\hat{\reg{A}}}\otimes \C^2_{\reg{A'}}$, with $X\simeq_{\sqrt{\eps}} \Id_{\hat{\reg{A}}}\otimes \sigma_X$, $Z\simeq_{\sqrt{\eps}} \Id_{\hat{\reg{A}}}\otimes\sigma_Z$. The remaining parts of the test certify that $A$ and $B$ are mapped to operators acting on $\hat{\reg{A}}$ only, while $C$ and $X_R$ both have a block form as specified in~\eqref{eq:def-xr}. Part (b) of the test ensures that the two blocks of $C$ relate to $A$ and $B$ as required, i.e. certifies the ``Moreover'' part of the lemma. 

Finally, success in $\comt(X_R,C)$ in part (a) certifies the commutation relation $[X_R,C]\approx 0$, which given the block decomposition in~\eqref{eq:def-xr} implies $RAR^\dagger = B$ (for some unitary $R$ specified by the lower left block of $X_R$). 
\end{proof}



%============================%
\section{The $n$-qubit Pauli group}
\label{sec:pauli-group}
%============================%

In this section we formulate a robust self-test for the $n$-qubit Pauli group. Even though the result is mostly known, it will help establish the framework we will use for the Clifford group in Section~\ref{sec:clifford}. In addition, we need to provide a means to explicitly account for a conjugation ambiguity in the definition of the Pauli observable $\sigma_Y$. 

%-----------------------------------%
\subsection{The $n$-qubit Weyl-Heisenberg group}
\label{sec:pbt}
%-----------------------------------den%

We start by giving a self-test for tensor products of $\sigma_X$ and $\sigma_Z$ 
observables acting on $n$ qubits, i.e. the $n$-qubit Weyl-Heisenberg group $\heisgn$. 
Let $\mathcal{P}^{(n)}$ denote the relations  
\begin{align*}
\paulin\{X,Z\} &= \Big\{ W(a)\in\Obs,\;W \in \prod_{i=1}^n \{X_i,Z_i\},\,a\in\{0,1\}^n\Big\} \\
&\qquad\cup \Big\{W(a)W'(a')=(-1)^{|\{i:\,W_i\neq W'_i \wedge a_ia'_i=1\}|} W'(a')W(a),\;\forall a,a'\in\{0,1\}^n\Big\}\\
&\qquad \cup\Big\{ W(a)W(a')=W(a+a'),\;\forall a,a'\in\{0,1\}^n\Big\}.
\end{align*}
Recall the notation $W(a)$ for the string that is $W_i$ when $a_i=1$ and $I$ otherwise. 
The first set of relations expresses the canonical anti-commutation relations. The second set of relations expresses that $X_i^2=Z_i^2=\Id$, coordinate-wise. It is easy to verify that $\mathcal{P}^{(n)}$ forms a defining set of relations for $\heisgn$. Our choice of relations is suggested by the Pauli braiding test introduced in~\cite{natarajan2016robust}, which shows that the relations are testable with a robustness parameter $\delta(\eps)$ that is independent of $n$. 
The underlying test is called the Pauli braiding test, and denoted $\pbt(X,Z)$. For convenience here we use a slight variant of the test, which includes more questions; the test is summarized in Figure~\ref{fig:pbt}. 

\begin{figure}[H]
\rule[1ex]{16.5cm}{0.5pt}\\
Test~$\pbt(X,Z)$: $X=X_1\cdots X_n$ and $Z=Z_1\cdots Z_n$ are two strings, where for each $i\in\{1,\ldots,n\}$, $X_i$ and $Z_i$ are anti-commuting observables on the $i$-th qudit. 
\begin{itemize}
\item Inputs: $(W,a)$, for $W\in\prod_{i=1}^n\{X_i,Z_i\}$ and $a\in\{0,1\}^n$.
\item Relations: $\paulin\{X,Z\}$.  
\item Test: Perform the following with probability $1/2$ each: 
\begin{enumerate}
\item[(a)] Select $W,W'\in \prod_i \{X_i,Z_i\}$, and $a,a'\in\{0,1\}^n$, uniformly at random. If $\{i: W_i\neq W'_i \wedge a_i=a'_i=1\}$ has even cardinality then execute test $\comt(W(a),W'(a'))$. Otherwise, execute test $\act(W(a),W'(a'))$. 
\item[(b)] Select $(a,b)\in\{0,1\}^n$ and $W\in\prod_{i=1}^n\{X_i,Z_i\}$ uniformly at random. Let $a,a'\in\{0,1\}^n$ be chosen uniformly at random. Execute test $\prodt(W(a),W(a'),W(a+a'))$. 
\end{enumerate}
\end{itemize}
\rule[2ex]{16.5cm}{0.5pt}\vspace{-1cm}
\caption{The Pauli braiding test, $\pbt(X,Z)$.}
\label{fig:pbt}
\end{figure}

\begin{lemma}[Theorem 13~\cite{natarajan2016robust}]
The test $\pbt(X,Z)$ is a robust $(1,\delta)$ self-test 
for $\paulin\{X,Z\}$, for some $\delta(\eps) = O(\eps^{1/2})$. 
\end{lemma}

In addition we need the following lemma, which states that observables approximately satisfying the relations $\paulin\{X,Z\}$ are close to operators which, up to a local isometry, behave exactly as a tensor product of Pauli $\sigma_X$ and $\sigma_Z$ observables. 


\begin{lemma}[Theorem 14~\cite{natarajan2016robust}]\label{lem:pauli-stable}
The set of relations $\mathcal{P}^{(n)}$ is $\delta$-stable, with $\delta(\eps) = O(\eps)$.
\end{lemma}



%-----------------------------------%
\subsection{The tensor product test}
\label{sec:perm}
%-----------------------------------%

Before we move on to provide a test for the full Pauli group, including not only $X,Z$ but also $Y$ observables, we use the Pauli braiding test introduced in the previous section to develop a test verifying that observables respect a certain tensor product structure created by $X$ and $Z$ observables. 

Let $k\geq 1$ be an integer and $S\subseteq \Obs((\C^2)^{\otimes k})$ a finite set of $k$-qubit observables (for us $k$ will always be a constant independent of $n$). We introduce a test for $(nk)$-qubit observables obtained as the $n$-fold tensor product of observables from $S$. 


\begin{figure}[H]
\rule[1ex]{16.5cm}{0.5pt}\\
Test~$\perm(S,n)$: $S\subseteq \Obs((\C^2)^{\otimes k})$ a finite set of observables, for some $k\geq 1$. Assume without loss of generality that $S$ contains $\{X,Z\}^{\otimes k}$. Assume $n$ even. \\
The verifier performs the following one-round interaction with two
players. Select $W \in S^n$ uniformly at random. With equal probability,
\begin{enumerate}
\item[(a)] Let $X = X_1\cdots X_{nk}$ and $Z=Z_1\cdots Z_{nk}$. Execute test $\pbt(X,Z)$ on $nk$ qubits. 
\item[(b)] For each $i\in\{1,\ldots,n\}$ let $W'_i\in\Obs((\C^2)^{\otimes k})$ be an observable that anti-commutes with $W_i$. Execute test $\pbt(W,W')$. 
\item[(c)] Select $W' \in S^n$. Send $W$ to one player and $W'$ to the other. The players reply with $a,a'\in\{\pm 1\}^n$ respectively. The verifier rejects if there exists an $i$ such that $W_i=W'_i$ and $a_i\neq a'_i$.
\item[(d)] Select a random permutation  $\sigma \in \mathfrak{S}_{n/2}$. Write $W=W_1 W_2$, where $W_1,W_2\in S^{n/2}$. Let $W_1^\sigma$ be the string $W_1$ with its entries permuted according to $\sigma$. Do the following with equal probability: 
\begin{enumerate}
\item[(i)] Send one player $W_1 W_1^\sigma$ and the other $W_1 W_2$ (resp. $W_2W_1^\sigma$), and check consistency of the first (resp. second) half of the players' answer bits.
\item[(ii)] Send one player $W_1  W_1^\sigma$, and the other $\prod_i \SWAP_{i,\sigma(i)}$, where each $\SWAP_{i,\sigma(i)}$ should be understood as acting on a chunk of $k$ consecutive qubits. 
    The first player replies with $a\in\{\pm 1 \}^n$, and the second with $b\in\{\pm1\}^{n/2}$,  corresponding to the outcomes obtained from applying each $\SWAP_{i,\sigma(i)}$ observable. For each
    $i\in\{1,\ldots,n/2\}$, check that $a_i a_{\sigma(i)} = b_i$. 
\item[(iii)] Execute $nk/2$ copies of test $\bellt$, for qubit pairs $(ki+t,kn/2+k\sigma(i)+t)$, for $i\in\{1,\ldots,n/2\}$ and $t\in\{0,\ldots,k-1\}$. 
\end{enumerate}
\end{enumerate}
\rule[2ex]{16.5cm}{0.5pt}\vspace{-1cm}
\caption{The permutation test, $\perm(S,n)$.}
\label{fig:clifford-test}
\end{figure}

\begin{lemma}\label{lem:perm-test}
Let $k\geq 1$, $S\subseteq \Obs((\C^2)^{\otimes k})$ a finite set of observables, and $n$ an even integer. Suppose given a strategy for the players that has success at least $1-\eps$ in test $\perm(S,n)$. Then there exists a local isometry $V_A:\mH_{\reg{A}}\to ((\C^2)^{\otimes k})_{\reg{A}_1}\otimes \cdots\otimes ((\C^2)^{\otimes k})_{\reg{A}_n}\otimes \hat{\mH}_{\hat{\reg{A}}}$, and observables $\sigma_{W} \in \Obs((\C^2)^{\otimes k}\otimes\hat{\mH})$ for $W\in S$ such that 
$$ \Es{W\in S^n} \big\|\big(W - (\prod_i \sigma_{W_i} ) V_A\big)\otimes \Id_{\reg{B}}\ket{\psi} \big\|^2 = O(\sqrt{\eps}),$$
where in the product, the $i$-th $\sigma_{W_i}$ acts on registers $\reg{A}_i$ and $\hat{\reg{A}}$; moreover, $\sigma_{W_i}$ acting on distinct $\reg{A}_i$, $\reg{A}_j$ ($i\neq j$) commute (even though they all act on $\hat{\reg{A}}$). 
Here the implied constant does not depend on $n$, but hides an exponential dependency in $k$.
\end{lemma}

%The main content of the lemma is in the tensor product form of $\sigma_{W} = \otimes_i \sigma_{W_i}$, with $\sigma_{W_i}$ acting on the $i$-th chunk of $k$ qubits and depending on $W_i$ only. Aside from this tensor product structure nothing else is shown about the observable $\sigma_{W_i}$, which may be arbitrary. 

\begin{proof}[Proof sketch.]
Using Lemma~\ref{lem:pauli-stable} and Lemma~\ref{lem:xyz-rigid}, part (a) of the test implies the existence of a local isometry under which $\mH \simeq (\otimes_i (\C^2)^{\otimes k}) \otimes \hat{\mH}$, and each $(nk)$-qubit Pauli operator $P\simeq_{\sqrt{\eps}} \sigma_P$, for $P\in\{I,X,Z\}^{nk}$. 

Using part (b) of the test and Lemma~\ref{lem:xyz-rigid}, it follows that an arbitrary $W\in S^n$ has a decomposition $W\simeq \prod_i \sigma_{W_i,i}$. Using the consistency checks and part (a) of the test, $\sigma_{W_i,i}$ (approximately) commutes with $\sigma_P$ for $P\in\{X,Z\}^{k(n-1)} \times I^k$ acting on all systems but the $i$-th; thus we may assume that $\sigma_{W_i,i}$ is supported on the $i$-th chunk of $k$ qubits and $\hat{\mH}$ only. Then $\sigma_{W_i,i}$ has an expansion 
$$ \sigma_{W_i,i} = \sum_{P\in\{I,X,Y,Z\}^k} \Id^{\otimes k(n-1)}\otimes  \sigma_{P,i} \otimes \Delta_{W_i,i}(P),$$
with $\sigma_{P,i}$ acting on the $i$-th chunk only. Using Claim~\ref{claim:swap-tt} below, part (d) of the test checks that for $i\neq j$ but $W_i=W_j$, 
$$\sum_P  \Delta_{W_i,i}(P) \Delta_{W_j,j}(P) \approx \Id_{\hat{\reg{A}}},$$
which gives that $\sigma_{W_i,i}$ depends on $W_i$ only, not on $i$. 
\end{proof}


\begin{claim}\label{claim:swap-tt}
Let $A\in \Obs(\C^2_{\reg{A}_1}\otimes \cdots\otimes \C^2_{\reg{A}_k})$ and $B\in \Obs(\C^2_{\reg{B}_1}\otimes \cdots\otimes \C^2_{\reg{B}_k})$ be $k$-qubit observables, and $\SWAP_{\reg{A'B'}} = \prod_{j=1}^k \SWAP_{\reg{A'}_j,\reg{B'}_j}$ the SWAP operator which permutes the $k$ $\reg{A'}_j$ registers with the $k$ $\reg{B'}_j$ registers.  Then 
\begin{equation}\label{eq:swap}
 \Big(\bigotimes_j \bra{\EPR}_{\reg{A}_j\reg{A'}_j}\bra{\EPR}_{\reg{B}_j\reg{B'}_j} \Big)\Big(\big(A_{\reg{A}}\otimes B _{\reg{B}}\big)\otimes \SWAP_{\reg{A'B'}}\Big) \Big(\bigotimes_j \ket{\EPR}_{\reg{A}_j\reg{A'}_j}\ket{\EPR}_{\reg{B}_j\reg{B'}_j} \Big) = \frac{1}{2^{k}}\Tr\big(A B\big).
\end{equation}
\end{claim}


\begin{proof}
We do the proof for $k=1$, as the general case is similar. Using 
$$\bra{\EPR}_{\reg{AA'}}\bra{\EPR}_{\reg{BB'}} \big( X_{\reg{AB}} \otimes Y_{\reg{A'B'}} \big) \ket{\EPR}_{\reg{AA'}}\ket{\EPR}_{\reg{BB'}} \,=\, \frac{1}{4}\Tr(XY^T),$$
the left-hand side of~\eqref{eq:swap} evaluates to $4^{-1} \Tr( (A\otimes B) \SWAP^T)$, which using the identity 
$$\SWAP_{\reg{A' B'}} = \sum_{a,b\in\{0,1\}} \ket{a}\bra{b}_\reg{A'}\ket{b}\bra{a}_{\reg{B'}}$$
gives the right-hand side of~\eqref{eq:swap}.
\end{proof}


%-----------------------------------%
\subsection{The $n$-qubit Pauli group}
\label{sec:e-pbt}
%-----------------------------------%

We will need an extended version of the Pauli braiding test introduced in the section~\label{sec:pbt} which also allows to test for a third observable, $Y_i$, on each system. Ideally we would like to enforce the relation $Y_i=\sqrt{-1}X_iZ_i$. Unfortunately, the complex phase cannot be tested from classical correlations alone: complex conjugation leaves correlations invariant, but does not correspond to a unitary change of basis  (see~\cite[Appendix A]{reichardt2012classicalarxiv} for an extended discussion of this issue). 

In our testing scenario the ``choice'' of complex phase, $\sqrt{-1}$ or its conjugate $-\sqrt{-1}$, will be represented by an observable $\Delta$ that the prover measures on a system that is in tensor product form from all other systems on which the prover acts. Informally, the outcome obtained when measuring $\Delta$ tells the prover to use $Y = i XZ$ or $Y=-iXZ$. 

We first introduce $Y$ and test that the triple $\{X,Y,Z\}$ pairwise anticommute at each site. This corresponds to the following set of relations: 
\begin{align*}
& {\epaulin}\{X,Y,Z\} = \Big\{ W(a)\in\Obs,\;W \in \{X,Y,Z\}^n,\,a\in\{0,1\}^n\Big\} \\
&\qquad\cup \Big\{W(a)W'(a')=(-1)^{|\{i:\,W_i\neq W'_i \wedge a_ia'_i=1\}|} W'(a')W(a),\;\forall a,a'\in\{0,1\}^n\Big\}\\
&\qquad \cup\Big\{ W(a)W(a')=W(a+a'),\;\forall a,a'\in\{0,1\}^n\Big\}.
\end{align*}




\begin{figure}[H]
\rule[1ex]{16.5cm}{0.5pt}\\
Test~$\pbt(X,Y,Z)$: $X$, $Y$ and $Z$ are three mutually anti-commuting observables on a single qudit. 
\begin{itemize}
\item Inputs: $W\in\prod_{i=1}^n\{X,Y,Z\}$
\item Relations: $\epaulin\{X,Y,Z\}$.  
\item Test: Perform the following with equal probability: 
\begin{enumerate}
\item[(a)] Execute test $\pbt(X^n,Z^n)$. 
\item[(b)] Select $W,W'\in  \{X,Y,Z\}^n$ uniformly at random such that $W'_i\neq W_i$ for each $i$. Execute test $\pbt(W,W')$. 
\item[(c)] Let $S = \{I,X,Y,Z\}$. Execute test $\perm(S,n)$. 
\end{enumerate}
\end{itemize}
\rule[2ex]{16.5cm}{0.5pt}\vspace{-1cm}
\caption{The extended Pauli braiding test, $\pbt(X,Y,Z)$.}
\label{fig:e-pbt}
\end{figure}


%
%\begin{lemma}
%\label{lem:ext-pbt-rigid-test}
%The test $\pbt(X,Z)$ is a robust $(1,\delta)$ self-test 
%for $\epaulin\{X,Y,Z\}$, for some $\delta(\eps) = O(\eps^{1/2})$. 
%\end{lemma}
%
%\begin{proof}
%The test $\pbt\{X,Y,Z\}$ is similar to $\pbt\{X,Z\}$, except in parts (a) and (b) the observables $W,W'$ are now chosen in $\prod_i\{X_i,Y_i,Z_i\}$ instead of $\prod_i\{X_i,Z_i\}$. 
%\end{proof}

Just as $\paulin\{X,Z\}$, the relations $\epaulin\{X,Y,Z\}$ are stable. However, by the aforementioned discussion these relations do not imply that $Y= iXZ$. Instead, the closest we can get is the following rigidity lemma. 

\begin{lemma}\label{lem:xyz-rigid}
Suppose $\ket{\psi}\in\mH_\reg{A}\otimes \mH_\reg{B}$ and $W(a) \in \Obs(\mH_\reg{A})$, for $W\in \{X,Y,Z\}^n$ and $a\in\{0,1\}^n$, specify a strategy for the players that has success probability at least $1-\eps$ in the extended Pauli braiding test $\pbt(X,Y,Z)$ described in Figure~\ref{fig:e-pbt}. 
Then there exists isometries $V_D:\mH_\reg{D} \to ((\C^2)^{\otimes n})_{\reg{D'}}  \otimes \hat{\mH}_{\hat{\reg{D}}}$, for $D\in\{A,B\}$, such that
$$\big\| (V_A \otimes V_B) \ket{\psi}_{\reg{AB}} - \ket{\EPR}_{\reg{A}'\reg{B}'}^{\otimes n} \ket{\aux}_{\hat{\reg{A}}\hat{\reg{B}}} \big\|^2 = O(\sqrt{\eps}),$$
and for $W\in \{X,Y,Z\}^n$,
\begin{align}\label{eq:test-sigmay}
 \Es{a\in\{0,1\}^n} \big\| \big(W(a) -V_A^\dagger (\sigma_W(a) \otimes \Delta_W(a)) V_A\big) \otimes \Id_B \ket{\psi} \big\|^2 &= O(\sqrt{\eps}),
\end{align}
where $\Delta_W(a) = \prod_i \Delta_{W_i}^{a_i} \in \Obs(\hat{\mH}_A)$ are observables with $\Delta_{X}=\Delta_{Z}=\Id$ and $\Delta_{Y}$ an arbitrary observable on $\hat{\mH}$ such that
	$$ \big\|\Delta_Y \otimes \Delta_Y \ket{\aux} - \ket{\aux} \big\|^2 = O(\sqrt{\eps}).$$
\end{lemma}

\begin{proof}[Proof sketch]
We first define the isometries $V_A$ and $V_B$ in the standard way (see e.g.~\cite[Theorem 2.3]{chao2017overlapping}) by using part (a) of the test. Under this isometry we have $X(a) \simeq \sigma_X(a)$ and $Z(b)\simeq \sigma_Z(b)$, for $a,b\in\{0,1\}^n$. Under the same isometry, for $c\in\{0,1\}^n$ the observable $Y(c)$ has an expansion 
$$ Y(c) \simeq \sum_{a,b} \sigma_X(a)\sigma_Z(b) \otimes \tilde{\Delta}_{a,b}(c),$$
where $\tilde{\Delta}_{a,b}(c)$ are abitrary Hermitian operators on $\hat{\mH}_{\hat{\reg{A}}}$. Using the 
 anti-commutation relations between $Y(c)$ and $X(a)$ and $Z(b)$ verified in part (b) of the test, all terms in the sum such that $a\neq b$ must vanish, and the expression simplifies to 
$$ Y(c) \simeq \sigma_Y(c) \otimes {\Delta}(c),$$
for some observable ${\Delta}(c)$ on $\hat{\mH}_A$. Using part (c) of the test and Lemma~\ref{lem:perm-test} we conclude that $\Delta(c) = \Delta_Y^{|c|}$ for some fixed observable $\Delta_Y$. 

The last condition in the lemma follows from the consistency relations, which imply that $X(a)\otimes X(a)$, $Z(b)\otimes Z(b)$ and $Y(c)\otimes Y(c)$ all approximately stabilize $\ket{\psi}$. 
\end{proof}




%\subsection{Commutation and anti-commutation}
%
%Commutation and anti-commutation relations between a pair of given observables are straightforward to enforce. For the first, we use a variant of the ``dummy question'' test of~\cite{}. For the second, any ``anti-commutation game'' (in the sense of~\cite{}), such as the Magic Square game, can be used. 
%
%\begin{lemma}[Commutation]\label{lem:commute-test}
%Suppose the set of relations $\mathcal{R}$ is $(1,\delta)$-testable, and let $G$ be a test. Let $V,W \in \Obs$ be two symbols such that, if both $V,W$ appear in $\mathcal{R}$ then there is a strategy achieving completeness in $G$ in which $VW=WV$. Then the set of relations $\mathcal{R}' = \mathcal{R} \cup \{\{VWVW=\Id,\,W\in\Obs\}$ is $(1,\delta')$-testable for some $\delta'(\eps)=O(\delta(\eps)+\sqrt{\eps})$. 
%\end{lemma}
%
%\begin{proof}[Proof sketch]
%Consider a test for $\mathcal{R}$. Devise a new test as follows. The verifier executes the test for $\mathcal{R}$ with probability $1/2$. With probability $1/2$, she selects one of the question pairs $(V,VW)$, $(W,VW)$, and sends one question to each player. Here $VW$ should be interpreted as the four-outcome POVM which simultaneously measures the observables $V$ and $W$. The players reply with one and two bits respectively, and the verifier checks the natural consistency condition.
%
%Completeness of the test follows by setting $VW$ to be a simultaneous joint measurement of $V$ and $W$, which is possible since $V$ and $W$ commute.
%
%For soundness, let $\{VW^{a,b}\}_{a,b\in\{0,1\}}$ be the player's POVM
  %associated to question $VW$ in a strategy with success at least $1-\eps$.
  %Assume without loss of generality that the measurement is projective. From
  %$VW$ we can define two commuting observables $V' = \sum_{a,b} (-1)^a VW^{a,b}$
  %and $W' = \sum_{a,b} (-1)^b VW^{a,b}$. Consistency of $V'$ (resp. $W'$) with $V$ (resp. $W$) follows from the second part of the test. For example, if the question pair $(V,VW)$ was selected, success with probability at least $1-\eps$ implies 
	%$$\sum_{a,b} \bra{\psi} V^a \otimes VW^{a,b} \ket{\psi} = 1-O(\eps),$$
	%which, using $\sum_a V^a = \sum_{a,b} VW^{a,b} = \Id$, in turn implies the desired consistency between $V$ and $V'$:
		%$$\sum_{a} \bra{\psi} (-1)^a V^a \otimes \sum_{b} (-1)^a VW^{a,b} \ket{\psi} = 1-O(\eps).$$
%\end{proof}
%
%\begin{lemma}[Anti-commutation]\label{lem:ac-test}
%Suppose the set of relations $\mathcal{R}$ is $(1,\delta)$-testable, and let $G$ be a test. Let $V,W \in \Obs$ be two symbols such that, if both $V,W$ appear in $\mathcal{R}$ then there is a strategy achieving completeness in $G$ in which $VW=-WV$. Then the set of relations $\mathcal{R}' = \mathcal{R} \cup \{\{VWVW=-\Id,\,W\in\Obs\}$ is $(1,\delta')$-testable for some $\delta'(\eps)=O(\delta(\eps)+\sqrt{\eps})$. 
%\end{lemma}
%
%\begin{proof}[Proof sketch]
%The proof is similar to Lemma~\ref{lem:commute-test}, except now the second part of the test can be replaced by the test $\pbt\{V,W\}$, which enforces anti-commutation. 
%\end{proof}

%=======================%
\section{Testing tensor products of Clifford observables}
\label{sec:clifford}
%=======================%

In this section we develop a test for $n$-fold tensor products of single-qubit or two-qubit Clifford observables. In Section~\ref{sec:n-clifford} we apply the Conjugation test from Section~\ref{sec:conj-test}  to test the relations that dictate how an arbitrary $n$-qubit Clifford unitary acts by conjugation on the Pauli matrices. In Section~\ref{sec:n-2-clifford} we specialize, and sharpen, the test to the case of unitaries that can be expressed as the $n$-fold tensor product of two-qubit Clifford observables.  


\subsection{Testing Clifford unitaries}
\label{sec:n-clifford}

Let $W$ be an $n$-qubit Clifford unitary. $W$ is characterized, up to phase, by its action by conjugation on the $n$-qubit Weyl-Heisenberg group. This action is described  by linear functions $h_S:\{0,1\}^n\times\{0,1\}^n \to \Z_4$ and $h_X,h_Z:\{0,1\}^n\times\{0,1\}^n \to \{0,1\}^n$ such that
\begin{equation}\label{eq:conj-cliff}
W \sigma_X(a)\sigma_Z(b) W^\dagger = i^{h_S(a,b)}\sigma_X(h_X(a,b))\sigma_Z(h_Z(a,b)),\qquad\forall a,b\in\{0,1\}^n.
\end{equation}
Since conjugation by $W$ preserves the Pauli anti-commutation relations it must be that $h_X(a,b)\cdot h_Z(a,b) = a\cdot b+h_S(a,b)\mod 2$.
 For any $a,b\in\{0,1\}^n$ define
\begin{equation}\label{eq:def-control-c}
A(a,b) = i^{a\cdot b}X(a)Z(b), \qquad B(a,b) = i^{a\cdot b}i^{h_S(a,b)}X(h_X(a,b))Z(h_Z(a,b)),
\end{equation}
where the additional phase $i^{a\cdot b}$ is introduced to ensure that $A(a,b)$ and $B(a,b)$ are observables. Let $C(a,b)=C_{A(a,b),B(a,b)}$ as in~\eqref{eq:def-xr}. 
%Each of $A$ and $B$ can be written in a unique way as a product of commuting single-qubit Pauli observables $\{X,Y,Z\}$. 
The Clifford conjugation test $\conjc(W)$ described in Figure~\ref{fig:conjugation-test-2} provides a test for the set of relations 
\begin{align*}
\conjr_{h_S,h_X,h_Z}\{W\} &= \paulin\{X,Y,Z\}  \cup \{W\in \Unitary\} \cup \{\Delta\in\Obs\}\\
&\qquad \cup \big\{ W X(a)Z(b)W^\dagger = (i\Delta)^{h_S(a,b)}X(h_X(a,b))Z(h_Z(a,b)),\,\forall a,b\in\{0,1\}^n\big\} \\
&\qquad \cup \big\{ \Delta X(a) = X(a)\Delta,\,\Delta Z(b)=Z(b)\Delta,\,\forall a,b\in\{0,1\}^n\big\}.
\end{align*}
Note the presence of the observable $\Delta$, which arises from the conjugation ambiguity in the definition of $Y$ (see Lemma~\ref{lem:xyz-rigid}). 
 
\begin{figure}[H]
\rule[1ex]{16.5cm}{0.5pt}\\
Test~\conjc(W): $W \in \mathcal{C}_n$ an $n$-qubit Clifford unitary. Let $h_S,h_X,h_Z$ be such that~\eqref{eq:conj-cliff} holds, and $A(a,b),B(a,b)$ the observables defined in~\eqref{eq:def-control-c}. \\
The verifier performs the following one-round interaction with two
players. With equal probability,
\begin{enumerate}
\item[(a)] Execute test $\pbt(X,Y,Z)$ on $(n+1)$ qubits, where the last qubit is called the ``control'' qubit;
%\item[(b)] Execute test $\perm(S,n)$ with $S = \{X,Y,Z\}$ (and $k=1$);
\item[(b)] Select $a,b\in\{0,1\}^n$ uniformly at random. Let $C(a,b)$ be the observable defined from $A(a,b)$ and $B(a,b)$ in~\eqref{eq:def-xr}, with the block structure specified by the control qubit. Execute test $\conj\{A(a,b),B(a,b),W\}$.
\item[(c)] Ask one prover to measure either $X_W$ (the same observable used in $\conj(W))$, or $W$, or $X(e_{2n+1})$, and the other to measure the observables $(W, X(e_{n+1}))$, where in all cases the $X$ acts on the control qubit (the $(n+1)$-st). Check that the outcomes are consistent.
\end{enumerate}
\rule[2ex]{16.5cm}{0.5pt}\vspace{-1cm}
\caption{The Clifford conjugation test, $\conjc(W)$.}
\label{fig:conjugation-test-2}
\end{figure}


\begin{lemma}\label{lem:cliff-conj}
Let $W$ be an $n$-qubit Clifford unitary and $h_S,h_X,h_Z$ such that~\eqref{eq:conj-cliff} holds. Suppose a strategy for the provers succeed with probability at least $1-\eps$ in test $\conjc(W)$. Let $V_D:\mH_\reg{D} \to ((\C^2)^{\otimes n})_{\reg{D'}}  \otimes \hat{\mH}_{\hat{\reg{D}}}$ be the isometries from Lemma~\ref{lem:xyz-rigid}.  Let $\sigma_W$ be an $n$-qubit Clifford observable satisfying~\eqref{eq:conj-cliff}, with $X$ and $Z$ replaced by $\sigma_X$ and $\sigma_Z$. Then there exists an observable $\Delta_W$ on $\hat{\reg{A}}$ such that $W \simeq_{\sqrt{\eps}} \sigma_W\otimes \Delta_W$, under the same isometry.  
\end{lemma}

Note that the observable $\sigma_W$ in the lemma is uniquely defined by~\eqref{eq:conj-cliff}, up to phase. The observable $\Delta_W$ takes the phase ambiguity into account.

\begin{proof}
 Completeness of the test is clear, as players making measurements on $(n+1)$ shared EPR pairs using standard Pauli observables, $W$, and $C(a,b)$ defined in~\eqref{eq:def-xr} with $A(a,b)$ and $B(a,b)$ as in~\eqref{eq:def-control-c} will pass all tests with probability $1$. 

Next we show soundness. The isometries $V_D$ follow from part (a) of the test and Lemma~\ref{lem:xyz-rigid}.
% success in part (a) of the test implies the existence of local isometries $V_A,V_B$ and observables $X(a,a')\approx_{\sqrt{\eps}} \sigma_X(a,a')$, $Y(c,c') \approx_{\sqrt{\eps}} \sigma_Y(c,c')\otimes \prod_i \Delta_i^{c_i}$ and $Z(b,b')\approx_{\sqrt{\eps}}\sigma_Z(b,b')$ for $a,b,c\in\{0,1\}^n$ and $a',b',c'\in\{0,1\}$. Using Lemma~\ref{lem:perm-test}, the permutation test from part (b) further implies that $\prod_i \Delta_i^{c_i} \approx_{\sqrt{\eps}} \Delta^{|c|}$ for some observable $\Delta$ independent of $i$. 
According to~\eqref{eq:def-control-c}, $A(a,b)$ and $B(a,b)$ can each be expressed (up to phase) as a tensor product of $X,Y,Z$ operators, where the number of occurrences of $Y$ modulo $2$ is $a\cdot b$ for $A(a,b)$ and $a\cdot b + h_S(a,b)$ for $B(a,b)$. Using consistency, we have, up to phase, $A(a,b) \simeq \sigma_X(a)\sigma_Z(b) \otimes \Delta_Y^{a\cdot b}$ and $B(a,b) \simeq \sigma_X(h_X(a,b)) \sigma_Z(h_Z(a,b)) \otimes \Delta_Y^{a\cdot b + h_S(a,b)}$, under the same isometry. Applying the analysis of the conjugation test given in Lemma~\ref{lem:conj} shows that $W$ conjugates $A(a,b)$ to $B(a,b)$, for all $a,b\in\{0,1\}^n$ (the  purpose of part (c) of the test is to ensure that the observable $X_W$ which appears in the relations $\mathcal{C}\{W,C(a,b)\}$ in is consistent with the observable $W$ used in this test). From $\sigma_W$ we can construct a basis by considering all $\sigma_W \sigma_X(a)\sigma_Z(b)$ for $a,b\in\{0,1\}^n$ . Expanding $W$ in this basis, $W \simeq \sum_{a,b} \sigma_W \sigma_X(a)\sigma_Z(b)\otimes \Delta_W(a,b)$ for arbitrary $\Delta_W(a,b)$. Using~\eqref{eq:conj-cliff} it follows that $\Delta_W(a,b) \approx 0$ for $(a,b)\neq (0,0)$, so $W \simeq \sigma_W \otimes \Delta_W$, where $\Delta_W$ must be an observable. By considering conjugation relations involving an odd number of Pauli $Y$, it follows that $\Delta_W$ (approximately) commutes with $\Delta_Y$. 
\end{proof}

\subsection{The $n$-fold two-qubit Clifford group}
\label{sec:n-2-clifford}

We turn to testing observables in the $n$-fold direct product of the two-qubit Clifford group $\cliffordgb$. Any such observable is of course a $2n$-qubit Clifford unitary, so that the test developed in the previous section applies. In this section we combine the test with the permutation test to ensure that the local choices of bases, and the conjugation ambiguities represented by the observable $\Delta_W$ from the previous section, are consistent across different tensor product observables (this is made precise in Theorem~\ref{thm:clifford-ntest} below). 

 Fix an arbitrary subset $\mathcal{G} \subseteq \cliffordgb \cap \Obs(\C^2\otimes \C^2)$ of two-qubit traceless Clifford observables.\footnote{The traceless condition is for convenience; we will see below where it is used.} An example is 
$$\mathcal{G} = \{IG,GI,IY,YI,IZ,ZI,CNOT\},$$
 which happens to be a generating set for the group $\cliffordgb$ (since $P = \frac{i-1}{\sqrt{2}} XG$).

\begin{figure}[H]
\rule[1ex]{16.5cm}{0.5pt}\\
Test~$\cliff(\mathcal{G},n)$: $\mathcal{G} \subseteq \cliffordgb \cap \Obs(\C^2\otimes \C^2)$; $n$ an even integer.  For any $G\in \mathcal{G}$, let $R=R(G)$ be a two-qubit unitary such that $RGR^\dagger = XI$ ($R$ exists since $G$ is assumed traceless). Let $\hat{\mathcal{G}}$ be the union of $\mathcal{G}$ and all such matrices $R(G)$ for $G\in\mathcal{G}$.\\
The verifier performs the following one-round interaction with two
players. Select $W \in \mathcal{G}^n$ uniformly at random. With equal probability,
\begin{enumerate}
%\item[(a)] Execute test $\pbt(X,Y,Z)$ on $2n+1$ qubits;
\item[(a)] Execute the test $\conjc(W)$, with $W$ interpreted as a $2n$-qubit Clifford and the $(2n+1)$-st qubit serving as the control qubit; 
\item[(b)] Execute the permutation test $\perm(S,n)$ where $S = \hat{\mathcal{G}}$ (and $k=2$);
\item[(c)] Let $C$ be the observable defined from $W$ and $X(\sum e_{2i+1})$ as in~\eqref{eq:def-xr}, and $R$ a unitary such that $R_i W_i R_i^\dagger = X(e_{2i+1})$ for each $i$.
Execute test $\conj\{R,C\}$.
\end{enumerate}
\rule[2ex]{16.5cm}{0.5pt}\vspace{-1cm}
\caption{The $2n$-qubit Clifford test, $\cliff(\mathcal{G},n)$.}
\label{fig:clifford-test-3}
\end{figure}

\begin{theorem}\label{thm:clifford-ntest}
Suppose a strategy for the players succeeds in test $\cliff(\mathcal{G},n)$ (Figure~\ref{fig:clifford-test-3}) with probability at least $1-\eps$. Then  for $D\in\{A,B\}$ there exists an isometry 
$$V_D: \mathcal{H}_\reg{D} \to (\C^2)^{\otimes (2n+1)}_{\reg{D}'} \otimes \hat{\mH}_{\hat{\reg{D}}}$$
such that 
\begin{equation}\label{eq:psi-epr}
\big\| (V_A \otimes V_B) \ket{\psi}_{\reg{AB}} - \ket{\EPR}_{\reg{A}'\reg{B}'}^{\otimes (2n+1)} \ket{\aux}_{\hat{\reg{A}}\hat{\reg{B}}} \big\|^2 = O(\sqrt{\eps}),
\end{equation}
and %for any $W \in \mathcal{G}$ there is a $\Delta_W \in \Obs(\hat{\mH})$ such that the $\Delta_W$ pairwise commute and 
\begin{equation}\label{eq:clifford-ntest-close}
\Es{W\in\mathcal{G}^n} \big\| \Id_A \otimes \big( V_B W_B - \sigma_{W} V_B\big)   \ket{\psi}_{\reg{AB}} \big\|^2 = O(\sqrt{\eps}).
\end{equation}
Here for $W\in\{I,X,Z\}^2\cap \mathcal{G}$, $\sigma_W$ are the regular Pauli matrices, and $\sigma_Y = i\sigma_X\sigma_Z \otimes \Delta$ as in Lemma~\ref{lem:xyz-rigid}, with $\Delta$ an observable on $\hat{\mH}$ such that  $\|(\Id_{\hat{\reg{A}}}\otimes \Delta - \Delta\otimes\Id_{\hat{\reg{B}}})\ket{\aux}\|=O(\sqrt{\eps})$. The remaining $\sigma_W$ are defined using their expansion on $\sigma_X,\sigma_Y$, and $\sigma_Z$, e.g. $\sigma_G = (\sigma_X \otimes \Id_{\hat{\reg{A}}} + \sigma_Y\otimes \Delta_{\hat{\reg{A}}})/\sqrt{2}$.
\end{theorem}

\begin{proof}[Proof sketch]
The existence of the isometry, as well as~\eqref{eq:psi-epr} and~\eqref{eq:clifford-ntest-close} for $W \in \{I,X,Y,Z\}^{2n}$, follows from the test $\pbt(X,Y,Z)$, executed as part of the clifford conjugation test from part (a), and Lemma~\ref{lem:xyz-rigid}. 
Using part (a) of the test and Lemma~\ref{lem:cliff-conj} it moreover follows that every $W \in \mathcal{G}^n$ is mapped under the same isometry to $W \simeq \sigma_W \otimes \Delta_{W}$, for some observable $\Delta_W$ on $\hat{\reg{A}}$ which may a priori depend on the whole string $W$. Comparing this expression with the one that results from part (c) of the test by Lemma~\ref{lem:perm-test}, we deduce that $\Delta_W = \prod_i \Delta_{W_i}$, for commuting observables $\Delta_{W_i}$ depending on $W_i$ only. 

 Part (d) of the test amounts to checking the relations $\sigma_R \sigma_W \sigma_R^\dagger \otimes \Delta_W \Delta_R^2 \approx \sigma_X$ for $W\in \mathcal{G}$. Since by definition $\sigma_R \sigma_W \sigma_R^\dagger = \sigma_X$, we obtain $\Delta_W \approx \Id$ for each $W\in\mathcal{G}$. 


%Let $W\in \mathcal{G}^n$ and $h_S,h_X,h_Z$ as in~\eqref{eq:conj-cliff}, and define 
%$$A_1(W) = \Es{a,b\in\{0,1\}^n} i^{-h_S(a,b)} \sigma_Z(h_Z(a,b))\sigma_X(h_X(a,b)) VWV^\dagger   \sigma_X(a)\sigma_Z(b).$$
%Then $A_1(W)$ is Hermitian and $\|A_1(W)\|\leq 1$. Part (b) of the test implies $ W \approx_{\sqrt{\eps}} A_1(W)$.
%For each $i\in\{1,\ldots,n\}$ we can complete $\sigma_{W_i}$ into a basis of anti-commuting Clifford observables for the $2$-qubit space on which $W_i$ acts; for example if $W_i = HI$ we can complete with $\{\sigma_{H'I},\sigma_{IH},\sigma_{IH'}\}$. Expanding $W$ in that $4^n$-element basis we find that $A_1(W) = \sigma_W \otimes \Delta_W$ for some Hermitian $\Delta_W$ of norm at most $1$ acting on $\hat{\mH}$.  
%
%Part (d) of the test implies the condition $\Delta_W \approx \prod_i \Delta_{i,W_i}$ for some $\Delta_{i,W_i}$. Consistency and commutation between different $\Delta_{i,W_i}$ follows from part (e) of the test. 
\end{proof}


We end with a corollary that expresses the conclusion~\eqref{eq:clifford-ntest-close} in terms of the post-measurement state of the first player. In addition to a set of two-qubit Clifford observables, our protocol requires us to make four-outcome measurements, such as a measurement in the Bell basis ($B$ measurement) or in the simultaneous eigenbasis of $Y\otimes X$ and $Z\otimes Z$ ($C$ measurement). Each of these can be expressed as the product of two commuting two-qubit Pauli observables, and can thus readily be tested via, e.g., the commutation test from Lemma~\ref{lem:elementary}. For convenience we include them in the corollary. 

For $W\in\mathcal{G}$ let $\sigma_W \in \Obs(\C^2 \otimes \C^2)$ be the associated observable, and $\sigma_W^0,\sigma_W^1$ the corresponding POVM elements. Similarly, if $W\in\{B,C\}$ we let $\sigma_W^a$, for $a\in\{0,1,2,3\}$, denote the POVM associated to $W$. 

\begin{corollary}\label{cor:clifford-rigid}
Let $\eps>0$, $n$ an even integer and $\mathcal{G} \subseteq (\cliffordgb \cap \Obs(\C^2 \otimes \C^2))\cup\{B,C\}$. Let $\hat{\mathcal{G}} = (\mathcal{G}\backslash\{B,C\}) \cup \{XX,ZZ,YX\}$. Suppose a strategy for the players succeeds with probability $1-\eps$ in test $\cliff(\hat{\mathcal{G}},n)$. Then for $D\in\{A,B\}$ there exists an isometry 
$$V_D: \mathcal{H}_\reg{D} \to (\C^2)^{\otimes 2n}_{\reg{D}'} \otimes \hat{\mH}_{\reg{D}}$$
such that
$$ \big\| (V_A \otimes V_B) \ket{\psi}_{\reg{AB}}  - \ket{\EPR}^{\otimes 2n}_{\reg{A}'\reg{B}'} \otimes \ket{\aux}_{\hat{\reg{A}}\hat{\reg{B}}} \big\|^2 = O(\sqrt{\eps}),$$
and (sub-normalized) densities $\tau_\epsilon$ for $\epsilon\in\{-1,1\}$ such that
$$ \Es{W\in\mathcal{G}^n} \sum_{u\in\prod_i \Lambda_i} \Big\| V_A \Tr_{\reg{B}}\big((\Id_A \otimes W^u) \proj{\psi}_{\reg{AB}} (\Id_A \otimes W^u)^\dagger\big) V_A^\dagger - \sum_{\epsilon\in\{-1,1\}} \Big( \otimes_{i=1}^n \frac{1}{4}\sigma_{W_i,\epsilon}^{u_i}\Big)\otimes \tau_\epsilon   \Big\|_1 = O(\sqrt{\eps}),$$
where $\Lambda_i = \{0,1\}$ if $W_i$ is a two-outcome observable and $\Lambda_i=\{0,1,2,3\}$ if $W_i$ is a four-outcome POVM; $\sigma_{X,\epsilon}=\sigma_X$, $\sigma_{Z,\epsilon} = \sigma_Z$, $\sigma_{Y,\epsilon} = \epsilon \sigma_Y$, and for the remaining $W\in\Sigma$, $\sigma_{W,\epsilon}$ is defined from $\sigma_{X,\epsilon},\sigma_{Y,\epsilon}$ and $\sigma_{Z,\epsilon}$ in the natural way. 
\end{corollary}

\begin{proof}
From Theorem~\ref{thm:clifford-ntest} we get isometries $V_A$, $V_B$ and an observable $\Delta$ on $\hat{\mH}$ such that the conclusions of the theorem hold. Let $\Delta = \Delta^{+1}-\Delta^{-1}$ be the eigendecomposition and 
$$\tau_{\epsilon} = \Tr_{\hat{\reg{B}}}\big( \big(\Id_{\hat{\reg{A}}}\otimes \Delta^\epsilon \big)\proj{\aux}\big(\Id_{\hat{\reg{A}}}\otimes \Delta^\epsilon\big)\big).$$
Each observable $W(b)$  can be expanded in terms of the projective measurement $\{W^u\}$ applied by the prover to define $W(b)$, as $W(b) = \sum_{u\in \prod \Lambda_i} (\prod_i \omega_i^{u_ib_i}) W^u$, where $\omega_i = e^{\frac{2i\pi}{|\Lambda_i|}}$ (recall this is how $W(b)$ is defined in the first place). Similarly, by definition $\sigma_W(b) = \otimes_i (\sum_{u_i\in\Lambda_i} \omega_i^{u_ib_i} \sigma_{W_i}^{u_i})$. Thus
\begin{align*}
\Es{b\in\prod_i \Lambda_i}\big\| \Id_A \otimes \big(  W_B(b) - V_B^\dagger \sigma_{W}(b) V_B\big)   \ket{\psi}_{\reg{AB}} \big\|^2
&= \Es{b\in\prod_i \Lambda_i}\Big\| \sum_u \big( \prod_i \omega_i^{u_ib_i}\big) \Id_A \otimes \big(  W_B^u - V_B^\dagger\sigma_{W}^u V_B\big)   \ket{\psi}_{\reg{AB}} \Big\|^2    \\
&= \sum_{u\in\prod_i \Lambda_i}\big\|  \Id_A \otimes \big(  W_B^u - V_B^\dagger\sigma_{W}^u V_B\big)   \ket{\psi}_{\reg{AB}} \big\|^2,       
\end{align*}
where the second line is obtained by expanding the square and using $\Es{b_i\in\Lambda_i} \omega_i^{u_ib_i} = 1$ if $u_i=0$, and $0$ otherwise. 
Using~\eqref{eq:psi-epr} and the form of $\sigma_W$ indicated in Theorem~\ref{thm:clifford-ntest},
\begin{align*}
    \sum_{u\in\prod_i \Lambda_i}\Big\|  \Tr_B \big( (\Id_A \otimes \sigma_{W}^u V_B)   \proj{\psi}_{\reg{AB}} (\Id_A \otimes \sigma_{W}^u V_B)^\dagger - \sum_{\epsilon\in\{-1,1\}} \Big( \otimes_{i=1}^n \frac{1}{4}\sigma_{W_i,\epsilon}^{u_i}\Big)\otimes \tau_\epsilon   \Big\|_1 = O(\sqrt{\eps}),
\end{align*}
where the factors $\frac{1}{4}$ correspond to the reduced density matrix of two EPR pairs, per observable $W_i$, on system $A$. 
\end{proof}





\subsection{Adaptive $n$-fold two-qubit Clifford group}
  Let $k_1, k_2, ..., k_l$ and $k_{< i} = \sum_{j < i} k_j$ and for $i > 1$, let $f_{i} :
  \mathcal{G}^{k_{1}} \times \{0,1\}^{k_{<i}}
  \rightarrow \mathcal{G}^{k_i}$. 

  We would like a  game that such that first we send $W \in \mathcal{G}^{k_1}$
  for the first Prover, who supposedly measures the first $k_1$ qubits according
  to the observables $W$ and has output $o_1 \in \{0,1\}^{k_1}$. The first
  Prover then measures the
  next $k_2$ qubits with the observables $f_2(W, o_1)$, with output $o_2 \in
  \{0,1\}^{k_2}$. The first Prover continues by measuring the next $k_3$ qubits
  with the observables $f_3(S, o_1o_2)$, and so on.

  Let $E_{W,o}$ be the set of observables that the first Prover should have
  applied when his input is $W$ and his output is $o = o_1,...,o_l$, i.e.,
  $E_{W,o} = W f_2(W,o_1) f_3(W,o_1o_2) .... f_l(W, o_1...o_{l-1})$.

  Ideally we want have a game such that we could conclude something similar to
  the next theorem.


\begin{theorem}
Suppose a strategy for the players succeeds in $\cliff(\mathcal{G},n)$ (Figure~\ref{fig:clifford-test-3}) with probability at least $1-\eps$. Then  for $D\in\{A,B\}$ there exists an isometry 
$$V_D: \mathcal{H}_\reg{D} \to (\C^2)^{\otimes (2n+1)}_{\reg{D}'} \otimes \hat{\mH}_\reg{D}$$
such that 
\begin{equation}
\big\| (V_A \otimes V_B) \ket{\psi}_{\reg{AB}} - \ket{\EPR}_{\reg{A}'\reg{B}'}^{\otimes (2n+1)} \ket{\aux}_{\hat{\reg{A}}\hat{\reg{B}}} \big\|^2 = O(\sqrt{\eps}),
\end{equation}
  and if the first Prover answers $o = o_1...o_l$ when asked $W \in
  \mathcal{G}^{k_1}$ then
\begin{equation}
  \Es{W\in\mathcal{G}^{k_1}} \Es{b\in\{0,1\}^n}\big\| \big( V_A {E_{W,o}}_A(b) -
  \sigma_{E_{W,o}}(b) V_A\big) \otimes I_B   \ket{\psi}_{\reg{AB}} \big\|^2 = O(\sqrt{\eps}).
\end{equation}
\end{theorem}

\tnote{This shouldn't be hard to prove. Two questions: can we say what the observables are in each iteration? Is it always the same set? Is it the case that in the honest strategy the outcomes $o_i$ should be uniformly distributed, so the choice of observable in each iteration should also be uniform (if we just consider the marginal distribution)? If this is the case maybe we can avoid having the $\eps$ add up with the number of rounds; I'm not sure}

\bibliography{delegation}

\notesendofpaper

\end{document}
