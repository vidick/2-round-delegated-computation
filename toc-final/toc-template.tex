%% toc-template.tex v0.4 2020-09-04 (last updated LB)
%% AUTHOR:
%% 1. Fill in the sections marked with triple exclamation marks !!!
%% 2. Place the following files in the same directory:
%%      toc.cls, tocbase.cls, tocplain.bst, and the article's source files
%% 3. Compile with "pdflatex"
%%      
%% 4. To include figures, uncomment "\usepackage{graphicx}" below, and use
%%    \begin{figure}[ht]
%%    \begin{center}
%%    \includegraphics[width=2in]{filename}
%%    \end{center}
%%    \caption{!!!}
%%    \label{fig:labeloffigure}
%%    \end{figure}
%%   Figures should be created in a _vector_ editing program, 
%%   such as the highly recommended IPE http://tclab.kaist.ac.kr/ipe/ 
%%   or the powerful language TikZ http://www.texample.net/tikz/.
%%   Do not use pixel-editing software like Paint or Photoshop,
%%     and also try to avoid PowerPoint. 
%%   We also discourage use of pstricks. 
%%   If for some reason your figures are in EPS format, convert to PDF
%%      using "epstopdf"

%!!! ToC#1XXX   %% replace 1XXX with the actual number -- example: ToC#1078

\documentclass{toc}

%% !!! AUTHOR: Fill in meta-data below
%% Optional items are marked %OPL, followed by explanation "if (when to use)"
%% if using optional item, delete "%OPL"
\tocdetails{%
  title = {full title of paper here},
%OPL  runningtitle = {short title to fit page headers}, %% if title long
%OPL  plaintexttitle = {title without LaTeX}, %% if title uses LaTeX
%
  number_of_pages = {NN}, 
  number_of_bibitems = {NN},
  number_of_figures = {NN},
  conference_version = {XXXXNN}, %% examples: {EC17},  {FSTTCS18}, {NONE}
    %% add "?" if submitted but no decision yet -- example: {CCC21?}
    %% please remember to notify editors if conf status changes
%OPL  conf_spec_issue,    %%  if conf spec issue article
%OPL  thematic_spec_issue = {title of spec issue},  %% if thematic spec issue  
%OPL  invited,   %% if paper invited by ToC editors but NOT spec issue
  author = {Author A, Author B, and Author C},
    %% Use the format
    %% "A", or "A and B", or "A, B, and C", or "A, B, C, and D", etc.
    %% e.g., {Author A, Author B, and K\'alm\'an Sz\H{o}l\H{o}ssy},
    %% IMPORTANT: Please use ascii TeX codes for characaters with
    %% foreign accents.

  authorlist = {Author A, Author B, Author C},
    %% Comma separated author list, NO AND: Use the format
    %% "A", or "A, B", or "A, B, C", etc.
    %% NOTE: No "and" at the end--simply comma separated,
    %% e.g., {Author A, Author B, K\'alm\'an Sz\H{o}l\H{o}ssy},
    %
    %% IF AUTHOR LIST LONG: add one or both optional entries below
    %% where the first name has been replaced with an initial.
%OPL  runningauthor = {A. Author, B. Author, and C. Author},
%OPL  copyrightauthor = {A. Author, B. Author, and C. Author},
  %% for some composite last names, the alphabetic position of
  %% the author's name is not the first letter of the last name
  %% for instance, Ronald de Wolf's last name is "de Wolf"
  %% but the alphabetic position of his last name is "W"
%OPL alphabetic_position = {N, X} %% example: {3, W} if 3rd author "de Wolf" 
  acmclassification = {ACM Classifications},
    %% Use the format "X.N.N, Y.N.N, ..., Z.N.N"; no period at end.
    %% See below for common ACM classifications.
  amsclassification = {AMS Classifications},
    %% Use the format "NNXNN, NNYNN, ..., NNZNN"; no period at end.
    %% See below for common AMS Classifications.
  keywords = {keyword, keyword, keyword},
    %% keywords and phrases of your choice, lower case, comma separated,
    %% no period at end
    %% please consider using relevant ToC categories -- see
    %%    http://theoryofcomputing.org/categories
    %    
}   %% END AUTHOR-FILLED METADATA

    %%% Common ToC ACM Classifications:
    %%% 
    %%% F.2.2 (Nonnumerical Algorithms and Problems)
    %%% F.1.3 (Complexity Measures and Classes)
    %%% G.3   (PROBABILITY AND STATISTICS)
    %%% F.1.2 (Modes of Computation)
    %%% G.2.2 (Graph Theory)
    %%% G.1.6 (Optimization)
    %%% F.2.3 (Tradeoffs between Complexity Measures)
    %%% F.2.1 (Numerical Algorithms and Problems)
    %%%
    %%% Common ToC AMS categories:
    %%% 68Q17 (Computational difficulty of problems (lower bounds, 
    %%% completeness, difficulty of approximation, etc.)
    %%% 68W25 (Approximation algorithms)
    %%% 81P68 (Quantum computation and quantum cryptography)
    %%% 68W20 (Randomized algorithms)
    %%% 68Q15 (Complexity classes (hierarchies, relations among 
    %%% complexity classes, etc.))
    %%% 68Q25 (Analysis of algorithms and problem complexity)
    %%% 68R10 (Graph theory)
    %%% 03F20 (Complexity of proofs)
    %%% 68Q32 (Computational learning theory)
    %%% 90C59 (Approximation methods and heuristics)
    %%% 
    %%% Full list available at:
    %%% http://www.acm.org/about/class/1998
    %%% http://www.ams.org/msc/
    %%% 

%%%%%%%%%%%%%%%%%%%%%%%%%%%%%%%%%%%%%%%%
%%% EDITOR: Fill in meta-data below
\tocdetails{%
%  volume = 1X,      %% example:  volume = 16,
%  number = Y,       %% examples: number = 5,    number = 19,
%  year = 20ZZ,
%  specissue={\cccBH}  %% CCC'17
%  specissue={\cccBI}  %% CCC'18
%  specissue={\cccBJ}  %% CCC'19
%  specissue={\cccCA}  %% CCC'20
%  specissue={\cccCB}  %% CCC'21
%  specissue={\approxrandomBG}  %% APPROX-RANDOM'16
%  specissue={\approxrandomBI}  %% APPROX-RANDOM'18
%  received = {date},
%  revised = {date},
%  published = {date},
%  note,
%  survey,  %% this item refers to "research survey", not "grad survey"
%  exposition, %% refers to "research exposition"
%  doi = {10.4086/toc....},  %% format: doi = {10.4086/toc.2018.v014a009}
}   %% END tocdetails

%%% !!!AUTHOR: add macro packages here
%%% Please add as few packages as possible
%%% Do not include the packages    amsmath, amsthm
%%% -- these are automatically loaded by the "toc" class
% \usepackage{graphicx}"   %%% !!! uncomment this if you include figures

%%% !!!AUTHOR put your marco definitions here
%%% PLEASE use as few of them as possible
%%% author-defined macros make copyediting more difficult

\begin{document}

\begin{frontmatter}%%[classification=text] << EDITOR.

%% EDITOR: If abstract fits entirely on first page, you may consider
%% the "classification=text" option, which typesets classifications
%% (as text) directly after the abstract--a preferable arrangement.

%%% !!! AUTHOR Title goes here
\title{title of paper here}  %% if conf version exists, see below

%%% !!! If a conference version exists, use instead the following
%%%   (after replacing the conference name)

%OPL  \title{TITLE OF PAPER\titlefootnote{XXX of this paper appeared in
%OPL  the \href{http://URL}{Proceedings of the 26th IEEE Conference on
%OPL  Computational Complexity, 2011}.}}
%OPL  preferable URL:  http://doi.org/[DOI]   %% if DOI available

%%% !!! Replace XXX by one of the following phrases: 
%%%   An extended abstract (if the current version adds or significantly 
%%%         expands the proofs of the main results stated in the conference
%%%         version but most of the main results of the current paper have 
%%%         already been (essentially) stated in the conference version);
%%%   A preliminary version (if the current version contains significant
%%%         new results or significant improvements over the results 
%%%         stated in the conference version); 
%%%   A conference version (in all other cases).
%%% Appropriately modify this text if the paper descends from more than one
%%% conference paper.   Make sure to include the conference version(s)
%%% in the .bib file.


%%% !!! AUTHOR List all authors. In brackets include the author's
%%% **last name** in lower case with no special characters; this
%%% will be used as the unique tag (author ID) to associate
%%% the author with the correct bio sketch at the end of the paper.

%%% List grant acknowledgements in \thanks.

\author[surname]{Firstname Surname\thanks{Supported by...}}
\author[szolossy]{K\'alm\'an Sz\H{o}l\H{o}ssy\thanks{Supported by...}}

%%% !!! AUTHOR Dedication, if desired, goes here  
%% \begin{dedication}
%% !!!
%% \end{dedication}

%%%  !!! AUTHOR Abstract goes here
%%%  limit your Abstract to 1920 characters to satisfy the arXiv standard
%%%  no \cite{...} commands in Abstract; citation format in abstract:
%%%   (Jones and Kumar, STOC'14)
%%%   if more than two authors: (Jones et al., STOC'14)
%%%   if journal: (Jones et al., SICOMP 2014)
\begin{abstract}
  !!!
\end{abstract}

\iffalse % DON'T TOUCH THIS LINE
%%%  AUTHOR: Choose the arXiv category that best fits your article.
%%%  A complete list of computer science categories is available here:
%%%     http://arxiv.org/archive/cs
%%%  and in math
%%%     http://arxiv.org/archive/math
%%%  Common categories include:
%%%     cs.CC - Computational Complexity
%%%     cs.CR - Cryptography and Security 
%%%     cs.DS - Data Structures and Algorithms
%%%     cs.LG - Learning
%%%     cs.IT - Information Theory 
%%%     cs.DM - Discrete Mathematics 
%%%     quant-ph - Quantum Physics
%%%     math.PR - Probability
%%%     math.CO - Combinatorics
%%%  You must include at least one category. If you include more than one category, 
%%%  the first one will serve as your main category, and the others will be used for
%%%  crosslisting. 
%%%
%%%  Example:
%%%  \tocarxivcategory{cs.CC,quant-ph}

\tocarxivcategory{!!!}

\fi % DON'T TOUCH THIS LINE

\end{frontmatter}

%%%
%%% !!! AUTHOR
%%% Paper goes here...


\bibliographystyle{tocplain}   %%% \bibliographystyle{plain}

%%% !!! AUTHOR
%%% Change this to match the name of your BIB file
\bibliography{template}

%%% !!! AUTHOR
%%% Include a short description of each author's affiliation
%%% following the structure below. Use the same unique ID used
%%% previously (presumably lower case last name).
%%% Use \tocat{} and \tocdot{} instead of "@" and "." in emails
\begin{tocauthors}
\begin{tocinfo}[surname]
 Firstname Surname\\
 Assistant professor\\
 Department of Computer Science and\\
 Department of Immunology\\
 Exemplar University\\
 Town, State/Province, Country\\
 name\tocat{}cs\tocdot{}exemplar\tocdot{}edu \\   %% email address here
 \url{http://cs.exemplar.edu/~surname}      %% your home page here
\end{tocinfo}
\begin{tocinfo}[szolossy]
  K\'alm\'an Sz\H{o}l\H{o}ssy\\
  Senior software engineer\\
  Logspace, Inc.\\
  Kecskem\'et, Hungary\\
  kszolos\tocat{}ailab\tocdot{}logspace\tocdot{}hu \\
  \url{http://www.logspace.hu/~kalman}
\end{tocinfo}
\end{tocauthors}

%%% !!!!
%%% Add a biographic sketch about each author.  The bio sections of
%%% previously published papers also appear in HTML, for example,
%%%   http://www.theoryofcomputing.org/articles/v003a009/about.html ).
%%% Include basic info about your education, research, and career
%%% (institutions, subject/title of dissertation, name of advisor, 
%%% list of areas of interest in some detail ["complexity theory" won't
%%% distinguish you from the majority of authors]).  Please also
%%% include information not readily available from other sources,
%%% like where you grew up, how you were first exposed to mathematics.
%%% The bio sketch is especially a good place to recognize an early
%%% mentor who helped turn your interest toward mathematics.  We also
%%% encourage you to include some more personal information (family,
%%% hobby, etc.).  Sprinkle it with humor.  IMPORTANT: Please include
%%% links in the bio sketch (to your advisor's and your Alma Mater's
%%% home page, your favorite hobby site, etc).  As far as links go,
%%% the more, the merrier. Use the following syntax:
%%%    Her advisor was \href{http://url}{name}.
%%% IMPORTANT: please remember to use ascii TeX codes for characters
%%% with a foreign accent.
\begin{tocaboutauthors}
\begin{tocabout}[surname]  %% use the same 
  \textsc{Firstname Surname}, known to her friends as <nickname>,
  graduated from
  \href{https://en.wikipedia.org/wiki/URL}{Paragon University} in
  1799; her advisor was
  \href{https://en.wikipedia.org/wiki/Johann_Friedrich_Pfaff}{Johann
    Friedrich Pfaff}. The subject of her thesis was integral rational
    algebraic functions.  ... (positions help, research interests)...
    Her attention was first turned to mathematics by her 7th grade
    piano teacher \href{http://url}{Sophia Maria Westenholz},
    a fan of math puzzles, with whom she spent ...  
    She enjoys hiking in the
   \href{https://en.wikipedia.org/wiki/Nilgiri_Mountains/}{Nilgiri Hills}.
    In her spare time she sometimes reads
  \href{http://theoryofcomputing.org}{\textsf{Theory of Computing}}.
\end{tocabout}
\begin{tocabout}[szolossy]
\textsc{K\'alm\'an Sz\H{o}l\H{o}ssy} is another fictitious author
whose name, with its copious accents, emphasizes the proper use
of TeX ascii codes.
\end{tocabout}
\end{tocaboutauthors}

\end{document}

