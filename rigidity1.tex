
Each of our delegation protocols includes a \emph{rigidity test} that is meant to verify that one of the provers measures his half of shared EPR pairs in a basis specified by the verifier, thereby preparing one of a specific family of post-measurement states on the other prover's space; the post-measurement states will form the basis for the delegated computation. 

The final rigidity test is given in Section~\ref{sec:n-2-clifford}; here we give a brief overview of the structure of the test. The test is parametrized by the number of EPR pairs to be used, which we will always take to be an even integer $2m$. For each consecutive pair of qubits $(2i,2i+1)$ the verifier chooses a basis  that can be specified as the simultaneous eigenbasis of a pair of commuting two-qubit observables $(B,B')$ that each belong to the two-qubit Clifford group $\mathcal{C}_2$. Even though it is not necessary, for simplicity we always assume that $(B,B')$ uniquely specify a basis of $\C^4$, so that there are no degeneracies. This includes, for example, $(B,B')=(XX,ZZ)$, which corresponds to measuring both qubits in the Bell basis, or $(B,B')=(XI,IZ)$, which corresponds to measuring the first qubit in the $X$ basis and the second in the $Z$ basis, etc. Let $\mathcal{G} = \{ (B,B') \} \subseteq (\mathcal{C}_2)^2$ denote the set of pairs of observables that are to be used as measurements in the test --- which observables are used depends on the requirements of the delegation protocol. We will always assume that $\mathcal{G}$ contains measurements in the Pauli basis on each qubit, i.e.\ \mbox{$\{XI,YI,ZI\}\times\{IX,IY,IZ\} \subseteq \mathcal{G}$}. 

Given an even integer $m$ and a choice of bases $\mathcal{G}$, the test $\ecliff(\mathcal{G},m)$ is a single-round classical interaction between the verifier and the two provers. With constant probability the verifier sends one of the provers a string $W$ chosen uniformly at random from $\mathcal{G}^m$; with the remaining probability other queries, requiring the measurement of observables not in $\mathcal{G}^m$, may be asked. (The honest strategy, which achieves completeness $1$, requires the provers to share $(2m+1)$ EPR pairs, and sometimes implement a joint observable on all their respective qubits.) %The test is designed such that any strategy that succeeds with probability $1-\eps$ overall is such that the measurement performed by the prover upon receipt of query $W \in \mathcal{G}^m$ is almost equivalent (up to local isometry and errors of order $O(\eps^{1/2})$) a measurement of each pair of qubits in the specified basis, with the prover's answer to the verifier consisting of the $m$ $2$-bit outcomes obtained. 

In general, an arbitrary strategy for the provers in the rigidity game consists of an arbitrary entangled state $\ket{\psi} \in \mH_\reg{A} \otimes \mH_\reg{B}$ (which we take to be pure), and measurements (which we take to be projective) for each possible question.\footnote{We make the assumption that the players employ a pure-state strategy for convenience, but it is easy to check that all proofs extend to the case of a mixed strategy. Moreover, it is always possible (and we always do) to consider projective strategies only by applying Naimark's dilation theorem, adding an auxiliary local system to each player as necessary, since no bound is assumed on the dimension of their systems.} This includes a $2m$-bit outcome projective measurement $\{W^u\}_{u\in\{0,1\}^{2m}}$ for each of the queries $W\in\mathcal{G}^m$. Our rigidity result states that any strategy that succeeds with probability $1-\eps$ in the test is within $\poly(\eps)$ of the honest strategy, up to local isometries (see Theorem~\ref{thm:clifford-ntest} for a precise statement). This is almost true, but for an irreconcilable ambiguity in the definition of the complex phase $\sqrt{-1}$. The fact that complex conjugation of observables 
leaves correlations invariant implies that no classical test can distinguish between the two nontrivial inequivalent irreducible representations of the Pauli group, which are given by the Pauli matrices $\sigma_X,\sigma_Y,\sigma_Z$ and their complex conjugates $\overline{\sigma_X}=\sigma_X$, $\overline{\sigma_Z}=\sigma_Z$, $\overline{\sigma_Y}=-\sigma_Y$ respectively. In particular, the provers may use a strategy that uses a combination of both representations; as long as they do so consistently no test will be able to detect this behavior.\footnote{See~\cite[Appendix A]{reichardt2012classicalarxiv} for an extended discussion of this issue, with a similar resolution to ours.}.  The formulation of our result accommodates this irreducible degree of freedom by forcing the provers to use a single qubit, the $(2m+1)$-st, to make their choice of representation. 


We introduce the language required to formulate our testing results in Section~\ref{sec:general-rigidity}. We then start by giving a test for the conjugation of one observable to another by a unitary, the Conjugation Test, in Section~\ref{sec:conj-test}. The test depends on a series of elementary tests, for commutation, anti-commutation, etc., that are presented in Appendix~\ref{sec:clifford-test}.  
In Section~\ref{sec:n-clifford}, we apply the Conjugation Test  to test the relations that dictate how an arbitrary $n$-qubit Clifford unitary acts by conjugation on the Pauli matrices. In Section~\ref{sec:n-2-clifford} we specialize, and sharpen, the test to the case of unitaries that can be expressed as the $n$-fold tensor product of two-qubit Clifford observables.  

%\begin{theorem}\label{thm:clifford-rigid}
%Let $\eps>0$, $n$ an integer and $\mathcal{G} \subseteq \cliffordgb\times\cliffordgb$ a finite set of pairs of commuting Clifford observables. Suppose a strategy for the players succeeds with probability $1-\eps$ in test $\ecliff(\mathcal{G},n)$. Then for $D\in\{A,B\}$ there exists an isometry 
%$$V_D: \mathcal{H}_\reg{D} \to (\C^2)^{\otimes 2n} \otimes {\mH}_{\hat{\reg{D}}}$$
%such that
%$$ \big\| (V_A \otimes V_B) \ket{\psi}_{\reg{AB}}  - \ket{\EPR}^{\otimes 2n} \otimes \ket{\aux}_{\hat{\reg{A}}\hat{\reg{B}}} \big\|^2 = O(\sqrt{\eps}),$$
%and orthogonal density matrices $\tau_\epsilon$ on $\hat{\reg{A}}$, for $\epsilon\in\{-1,1\}$, such that
%$$ \mathop{\textsc{E}}_{W\in\mathcal{G}^n} \sum_{u\in\{\pm 1\}^{2n}} \Big\| V_A \Tr_{\reg{B}}\big((\Id_A \otimes W_{\reg{B}}^u) \proj{\psi}_{\reg{AB}} (\Id_A \otimes W_{\reg{B}}^u)^\dagger\big) V_A^\dagger - \sum_{\epsilon\in\{-1,1\}} \Big( \bigotimes_{i=1}^n \frac{\sigma_{W_{2i},W_{2i+1},\epsilon}^{u_{2i},u_{2i+1}}}{4}\Big)\otimes \frac{\tau_\epsilon}{2}   \Big\|_1 = O(\eps^c).$$
%\end{theorem}
%
%It is important to observe that the post-measurement state on Alice's $2n$ EPR pairs is consistent, in the sense that the same basis (conjugate or not conjugate) is used for each of the $n$ pairs of qubits. In the delegation protocol the whole computation will take place on these $2n$ qubits, so that the choice of $\epsilon$ can be treated as a coin flip shared between the two provers. 
%
%Theorem~\ref{thm:clifford-rigid} provides a guarantee on the post-measurement state of prover $A$, conditioned on the reported outcomes of a measurement performed by prover $B$. If prover $A$ later claims to measure his $2n$ qubits in an arbitrary choice of basis (also chosen from $\mathcal{G}^n$), and reports the claimed outcomes, we will be able to check those outcomes that are obtained when the choice of basis made by prover $A$ happens to match the random choice of measurement that $B$ was instructed to perform. Since, moreover, $A$ is not told what measurement $B$ performed, for any choice of basis by $A$ there will be a constant probability with which $B$'s basis matches that choice. We formulate a test that models this scenario, test $\tom(\hat{\mathcal{G}},n)$ described in Section~\ref{}. In the test, with constant probability one of the players is not sent any question, but instead asked to report a choice of observables $W'\in\mathcal{G}^n$, and associated outcomes $u\in\{\pm 1 \}^{2n}$; we denote $\{Q^{W',u}\}$ the associated POVM. The following corollary states the results of the analysis of the test. The corrollary follows almost immediately from Theorem~\ref{thm:clifford-rigid} (see Section~\ref{} for the proof); it will be useful in our analysis of the Dog-Walker Protocol, presented in Section \ref{sec:dog-walker} 
%
%\begin{corollary}\label{cor:clifford-rigid-adaptive}
%Let $\eps>0$, $n$ an even integer and  $\mathcal{G} \subseteq \cliffordgb\times\cliffordgb$ a finite set of pairs of commuting Clifford observables. Suppose a strategy for the players succeeds with probability $1-\eps$ in test $\tom(\hat{\mathcal{G}},n)$. Let $V_A,V_B$ be the isometries specified in Theorem~\ref{thm:clifford-rigid}. Then there exists a distribution $q$ on ${\mathcal{G}}^n \times \{\pm 1\}$ such that 
%\begin{align*}
 %\sum_{W'\in\hat{\mathcal{G}}^n} \sum_{u\in \{\pm 1\}^{2n}} \Big\| \Tr_{\reg{A}\hat{\reg{B}}} &\big((\Id_A \otimes V_B Q^{W',u} )\proj{\psi}_{\reg{AB}} (\Id_A \otimes V_B Q^{W',u})^\dagger\big)\\
%&\hskip3cm - \sum_{\epsilon\in\{-1,1\}}  q(W',\epsilon)  \Big( \bigotimes_{i=1}^n \frac{\sigma_{W'_{2i},W'_{2i+1},\epsilon}^{u_{2i},u_{2i+1}}}{4}\Big) \Big\|_1 = O(\sqrt{\eps}).
%\end{align*}
%\end{corollary}


%============================%
\subsection{Testing}
\label{sec:general-rigidity}
%============================%

In this section we recall the standard formalism from self-testing, including state-dependent distance measure, local isometries, etc. We also introduce a framework of ``tests for relations'' that will be convenient to formulate our results. 


\paragraph{Distance measures.}
Ultimately our goal is to test that a player implements a certain tensor product of single-qubit or two-qubit measurements defined by observables such as $\sigma_X$, $\sigma_Y$, or $\sigma_G$. Since it is impossible to detect whether a player applies a certain operation $X$ on state $\ket{\psi}$, or $VXV^\dagger$ on state $V\ket{\psi}$, for any isometry $V:\Lin(\mH)\to\Lin(\mH')$ such that $V^\dagger V = \Id$, we will (as is standard in testing) focus on testing identity up to \emph{local isometries}. Towards this, we introduce the following important piece of notation: 

\begin{definition}
For finite-dimensional Hilbert spaces $\mH_{\reg{A}}$ and $\mH_{\reg{A'}}$, $\delta>0$, and operators $R \in\Lin(\mH_{\reg{A}})$ and $S\in\Lin(\mH_{\reg{A'}})$ we say that $R$ and $S$ are $\delta$-isometric with respect to $\ket{\psi} \in \mH_{\reg{A}} \otimes \mH_{\reg{B}}$, and write $R\simeq_\delta S$, if there exists an isometry $V:\mH_{\reg{A}}\to\mH_{\reg{A'}}$ such that 
$$\big\|( R-V^\dagger SV)\otimes \Id_{\reg{B}} \ket{\psi}\big\|^2=O(\delta).$$
If $V$ is the identity, then we further say that $R$ and $S$ are $\delta$-equivalent, and write $R\approx_\delta S$ for $\| ( R- S) \otimes \Id_{\reg{B}} \ket{\psi}\|^2=O(\delta)$.
\end{definition}

The notation $R\simeq_\delta S$ carries some ambiguity, as it does not specify the state $\ket{\psi}$. The latter should always be clear from context: we will often simply write that $R$ and $S$ are $\delta$-isometric, without explicitly specifying $\ket{\psi}$ or the isometry. The relation is transitive, but not reflexive: the operator on the right will always act on a space of dimension at least as large as that on which the operator on the left acts. The notion of $\delta$-equivalence is both transitive (its square root obeys the triangle inequality) and reflexive, and we will use it as our main notion of distance. 

\paragraph{Tests.}
We formulate our tests as two-player games in which both players are treated symmetrically.  We often use the same symbol, a capital letter $X,Z,W,\ldots,$ to denote a question in the game and the associated projective measurement $\{W^a\}$ applied by the player upon receipt of that question. To a projective measurement with outcomes in $\{0,1\}^n$ we  associate a family of observables $W(u)$ parametrized by $n$-bit strings $u\in\{0,1\}^n$, defined by $W(u) = \sum_a (-1)^{u\cdot a} W^a$. If $n=1$ we simply write $W=W(1)=W^0-W^1$; note that $W(0)=\Id$.

The games we consider always implicitly include a ``consistency test'' which is meant to enforce that whenever both players are sent identical questions, they produce matching answers. More precisely, let $T$ be any of the two-player tests described in the paper. Let $\Pr_T(W,W')$ be the distribution on questions $(W,W')$ to the players that is specified by $T$. Since the players are always treated symmetrically, $\Pr_T(\cdot,\cdot)$ is permutation-invariant. Let $\Pr_T(\cdot)$ denote the marginal on either player. Then instead of executing the test $T$ as described, the verifier performs the following:
\begin{enumerate}
\item[(i)] With probability $1/2$, execute $T$.
\item[(ii)] With probability $1/2$, select a random question $W$ according to $\Pr_T(W)$. Send $W$ to both players. Accept if and only if the players' answers are equal. 
\end{enumerate}
Then success with probability at least $1-\eps$ in the modified test implies success with probability at least $1-2\eps$ in the original test, as well as in the consistency test. If $\{W_{\reg{A}}^a\}$ and $\{W_{\reg{B}}^b\}$ are the players' corresponding projective measurements, the latter condition implies 
\begin{align}
\sum_a \|(W_{\reg{A}}^a \otimes \Id - \Id \otimes W_{\reg{B}}^a)\ket{\psi}_{\reg{AB}}\|^2 &= 2-2 \sum_a \bra{\psi} W_{\reg{A}}^a \otimes W_{\reg{B}}^a \ket{\psi} \notag\\ 
&\leq 4 \eps,\label{eq:consistency}
\end{align}
so that $W_{\reg{A}}^a \otimes \Id \approx_{\sqrt{\eps}} \Id \otimes W_{\reg{b}}^a$ (where the condition should be interpreted on average over the choice of a question $W$ distributed as in the test). Similarly, if $W_{\reg{A}}$, $W_{\reg{B}}$ are observables for the players that succeed in the consistency test with probability $1-2\eps$ we obtain $W_{\reg{A}}\otimes \Id \approx_{\sqrt{\eps}} \Id \otimes W_{\reg{B}}$. We will often use both relations to ``switch'' operators from one player's space to the other's; as a result we will also often omit an explicit specification of which player's space an observable is applied to. 

\paragraph{Strategies.} Given a two-player game, or test, a strategy for the players consists of a bipartite entangled state $\ket{\psi} \in \mH_\reg{A} \otimes \mH_\reg{B}$ together with families of projective  measurements $\{W^a_\reg{A}\}$ for Alice and $\{W_\reg{B}^a\}$ for Bob, one for each question $W$ that can be sent to either player in the test. As already mentioned, for convenience we restrict our attention to pure-state strategies employing projective measurements. 
%The fact that both players will always be treated symmetrically allows us to assume without loss of generality that $\mH_\reg{A} = \mH_\reg{B}$, $\ket{\psi}$ is permutation-invariant, and for any question $W$, $W_\reg{A}^a = W_\reg{B}^a$ for all answers $a$ (see e.g. Lemma~4 in~\cite{kempe2011entangled}).
We will loosely refer to a strategy for the players as $(W,\ket{\psi})$, with the symbol $W$ referring to the complete set of projective measurements used by the players in the game; taking advantage of  symmetry we often omit the subscript $\reg{A}$ or $\reg{B}$, as all statements involving observables for one player hold verbatim with the other player's observables as well. 

\paragraph{Relations.}
We use $\mathcal{R}$ to denote a set of relations over variables $X,Z,W,\ldots,$ such as
$$\mathcal{R}=\big\{XZXZ=-\Id,\, HX=ZH,\,X,Z,H\in\Obs\big\}.$$
We only consider relations that can be brought in the form either $f(W) = (-1)^a W_1\cdots W_k = \Id$, where the $W_i$ are (not necessarily distinct) unitary variables and $a\in\Z_2$, or $f(W) = W_1 \cdot ( \sum_a \omega_a W_2^a)=\Id$, where $W_1$ is a unitary variable, $\{W_2^a\}$ a projective measurement with $s$ possible outcomes, and $\omega_a$ are (arbitrary) $s$-th roots of unity.   

\begin{definition}[Rigid self-test]
We say that a set of relations $\mathcal{R}$ is $(c,\delta(\eps))$-testable, on the average under distribution $\mathcal{D}:\mathcal{R}\to[0,1]$, if
  there exists a game (or test) $G$ with question set $\mathcal{Q}$ that
  includes (at least) a symbol for each variable in $\mathcal{R}$ that is either an observable or a POVM and such that:
\begin{itemize}
\item (\emph{Completeness}) There exists a set of operators which exactly satisfy all relations in $\mathcal{R}$ and a strategy for the players which uses these operators (together possibly with others for the additional questions) that has success probability at least $c$;
\item (\emph{Soundness}) For any $\eps>0$ and any strategy $(W,\ket{\psi}_{AB})$ that succeeds in the game with probability at least $c-\eps$, the associated measurement operators satisfy the relations in $\mathcal{R}$ up to $\delta(\eps)$, in the state-dependent norm. More precisely, on average
 over the choice of a relation $f(W)=\Id$ from $\mathcal{R}$ chosen according to $\mathcal{D}$, it holds that $\| \Id\otimes (f(W)-\Id) \ket{\psi}_{\reg{AB}}\|^2 \leq \delta(\eps)$.
\end{itemize}
If both conditions hold, we also say that the game $G$ is a robust $(c,s)$ self-test for the relations $\mathcal{R}$. 
\end{definition}

Even though our definition is general, in this paper we only consider games with perfect completeness, $c=1$, so that we usually omit the parameter. The distribution $\mathcal{D}$ will often be implicit from context, and we do not always specify it explicitly (e.g. in case we only measure $\delta(\eps)$ up to multiplicative factors of order $|\mathcal{R}|$ the exact distribution $\mathcal{D}$ does not matter as long as it has complete support). 

\begin{definition}[Stable relations]
We say that a set of relations $\mathcal{R}$ is $\delta(\eps)$-stable, on the average under distribution $\mathcal{D}:\mathcal{R}\to[0,1]$, if for any two families of operators $W_A\in\Lin(\mH_\reg{A})$ and  $W_B\in\Lin(\mH_\reg{B})$ that are consistent on average, i.e. 
$$\Es{f \sim\mathcal{D}} \Es{W \in_U f} \big\| (\Id\otimes W_B - W_A \otimes \Id)\ket{\psi}\big\|^2\leq \eps,$$
where $W \in_U f$ is shorthand for $W$ being a uniformly random operator among those appearing in the relation specified by $f$,
and satisfy the relations on average, i.e. 
$$\Es{\substack{f\sim\mathcal{D}:\\f(W)=\Id \in\mathcal{R}}} \big\|  (f(W_A)- \Id) \otimes \Id \ket{\psi}\big\|^2 \leq \eps,$$
  there exists operators $\hat{W}$ which satisfy the same relations exactly and are $\delta(\eps)$-isometric to the $W$ with respect to $\ket{\psi}$, on the average over the choice of a random relation in $\mathcal{R}$ and a uniformly random $W$ appearing in the relation, i.e. there exists an isometry $V_A$ such that 
  \begin{equation}
    \Es{f \sim\mathcal{D}} \Es{W \in_U f} \big\|( \hat{W}_A-V_A^\dagger W_AV_A)\otimes \Id \ket{\psi}\big\|^2=O(\delta(\eps)).\nonumber
  \end{equation}
	\end{definition}


%---------------------------%
\subsection{The conjugation test}
\label{sec:conj-test}
%---------------------------%

We give a test which certifies that a unitary (not necessarily an observable) conjugates an observable to another. More precisely, let $A,B$ be observables, and $R$ a unitary, acting on the same space $\mH$. The test $\conj(R,C)$ certifies that the players implement observables of the form
\begin{equation}\label{eq:def-xr}
X_R = \begin{pmatrix} 0 & R^\dagger\\ R & 0 \end{pmatrix}\qquad \text{and}\qquad C = C_{A,B} = \begin{pmatrix} A & 0\\ 0 & B \end{pmatrix}
\end{equation}
such that $X_R$ and $C$ commute. The fact that $X_R$ is an observable implies that $R$ is unitary,\footnote{Note that $R$ will not be directly accessed in the test, since by itself it does not necessarily correspond to a measurement.} while the commutation condition is equivalent to the relation $RAR^\dagger = B$. The test thus tests for the relations
\begin{align*}
 \mathcal{C}\{R,C\} &= \big\{ X_R,C,X,Z\in \Obs\big\} \cup \big\{XZ=-ZX\big\}
\cup \big\{ X_R C = C X_R,\, X_RZ=-Z X_R,\, C Z=ZC\big\}.
\end{align*}
Here the anti-commuting observables $X$ and $Z$ are used to specify a basis in which $X_R$ and $C$ can be block-diagonalized. The anti-commutation and commutation relations with $Z$ enforce that $X_R$ and $C$ respectively have the form described in~\eqref{eq:def-xr}.
These relations are enforced using simple commutation and anti-commutation tests that are standard in the literature on self-testing. For convenience, we state those tests in Appendix~\ref{sec:clifford-test}. The conjugation test, which uses them as sub-tests, is given in Figure~\ref{fig:conjugation-test-1}. Here, ``Inputs'' refers to a subset of designated questions in the test; ``Relation'' indicates a relation that the test aims to certify; ``Test'' describes the certification protocol. (Recall that all our protocols implicitly include a ``consistency'' test in which a question is chosen uniformly at random from the marginal distribution and sent to both players, whose answers are accepted if and only if they are equal.)

\begin{figure}[H]
\rule[1ex]{16.5cm}{0.5pt}\\
Test~\conj(A,B,R) 
\begin{itemize}
    \item Inputs: $A$ and $B$ observables on the same space $\mH$, and $X$ and $Z$ observables on $\mH'$. $X_R$ and $C$ observables on $\mH\otimes \mH'$.
    \item Relations:  $\mathcal{C}\{R,C\} $, with $R$ defined from $X_R$, and $C$ related to $A$ and $B$, as in~\eqref{eq:def-xr}. 
    \item Test: execute each of the following with equal probability
		\begin{enumerate}
\item[(a)] With probability $1/8$ each, execute tests $\act(X,Z)$,  $\comt(C,Z)$, $\comt(X_R,C)$,   $\act(X_R,Z)$ and $\comt(A,X)$, $\comt(B,X)$, $\comt(A,Z)$, $\comt(B,Z)$. 
\item[(b)] Ask one player to measure $A$, $B$, $C$ or $Z$ (with probability $1/4$ each), and the other to jointly measure $A$ or $B$ (with probability $1/2$ each) and $Z$. The first player returns one bit, and the second two bits. Reject if either:
\begin{itemize}
\item The first player was asked $C$, the second player was asked $(A,Z)$, his second answer bit is $0$, and his first answer bit does not match the first player's;
\item The first player was asked $C$, the second player was asked $(B,Z)$, his second answer bit is $1$, and his first answer bit does not match the first player's.
\item The first player was asked $A$, $B$, or $Z$ and his answer bit does not match the corresponding answer from the second player.
\end{itemize}
\end{enumerate}
\end{itemize}
\rule[2ex]{16.5cm}{0.5pt}\vspace{-0cm}
\caption{The conjugation test, $\conj(A,B,R)$.}
\label{fig:conjugation-test-1}
\end{figure}

\begin{lemma}\label{lem:conj}
The test $\conj(A,B,R)$ is a $(1,\delta)$ self-test for the set of relations $\mathcal{C}\{R,C\}$, for some $\delta = O(\sqrt{\eps})$. Moreover, for any strategy that succeeds with probability at least $1-\eps$ in the test it holds that $C \approx_{\delta} A (\Id+Z)/2 + B(\Id-Z)/2$, where $A,B,C$ and $Z$ are the observables applied by the prover on receipt of a question with the same label. 
\end{lemma}

\begin{proof}
Completeness is clear, as players making measurements on a maximally entangled state on $\mH_{\reg{A}}\otimes \mH_{\reg{B}}$, tensored with an EPR pair on $\C^2 \otimes \C^2$ for the $X$ and $Z$ observables, and using $X_R$ and $C$ defined in~\eqref{eq:def-xr} (with the blocks specified by the space associated with each player's half-EPR pair) succeed in each test with probability $1$. 

We now consider soundness. Success in $\act(X,Z)$ in part (a) of the test implies the existence of local isometries $V_A,V_B$ such that $V_A:\mH_\reg{A}\to \mH_{\hat{\reg{A}}}\otimes \C^2_{\reg{A'}}$, with $X\simeq_{\sqrt{\eps}} \Id_{\hat{\reg{A}}}\otimes \sigma_X$ and $Z\simeq_{\sqrt{\eps}} \Id_{\hat{\reg{A}}}\otimes\sigma_Z$. By Lemma~\ref{lem:pauli-c}, approximate commutation with both $X$ and $Z$ implies that under the same isometry, $A\simeq_{\sqrt{\eps}} A_I \otimes \Id$ and $B\simeq_{\sqrt{\eps}} B_I \otimes \Id$, for observables $A_I, B_I$ on $\mH_{\hat{\reg{A}}}$. Similarly, the parts of the test involving $C$ and $X_R$ imply that they each have the block decomposition specified in~\eqref{eq:def-xr}. 

Next we analyze part (b) of the test. Let $\{W_{AZ}^{a,z}\}$ be the projective measurement applied by the second player upon query $(A,Z)$. Success with probability $1-O(\eps)$ in the first item ensures that 
$$\big| \bra{\psi} C \otimes (W_{AZ}^{00} - W_{AZ}^{10})\ket{\psi} \big| \,=\,O(\eps),$$
and a similar condition holds from the second item, with $W_{BZ}$ instead of $W_{AZ}$. Success with probability $1-O(\eps)$ in the third item ensures consistency of $\{W_{AZ}^{a,z}\}$ (resp. $\{W_{BZ}^{a,z}\}$) with the observable $A$ (resp. $B$) when marginalizing over the second outcome, and $Z$ when marginalizing over the second outcome. Using the decompositions for $A,B$ and $C$ derived from part (a) of the test, we obtain the ``Moreover'' part of the lemma. 

Finally, success in test $\comt(X_R,C)$ from part (a) certifies the approximate commutation relation $[X_R,C]\approx_{\sqrt{\eps}} 0$, which given the block decomposition in~\eqref{eq:def-xr} implies $RAR^\dagger \approx B$, where we wrote $X_R \simeq R_X \otimes \sigma_X + R_Y \otimes \sigma_Y$, and use that $X_R$ is an observable to deduce that there exists a unitary $R$ on $\mH_{\hat{\reg{A}}}$ such that $R \approx R_X + i R_Y$. 
\end{proof}



\subsection{Testing Clifford unitaries}
\label{sec:n-clifford}

\tnote{there are still inconsistencies between $n$ and $m$. This should all be fixed to $2m+1$ being the total number of qubits, so $n\to 2m+1$ in this section. But maybe we can just use $m$, and call the test with $2m+1$ later.}
Let $R$ be an $n$-qubit Clifford unitary. $R$ is characterized, up to phase, by its action by conjugation on the $n$-qubit Weyl-Heisenberg group. This action is described  by linear functions $h_S:\{0,1\}^n\times\{0,1\}^n \to \Z_4$ and $h_X,h_Z:\{0,1\}^n\times\{0,1\}^n \to \{0,1\}^n$ such that
\begin{equation}\label{eq:conj-cliff}
R \sigma_X(a)\sigma_Z(b) R^\dagger = (-1)^{h_S(a,b)}\sigma_X(h_X(a,b))\sigma_Z(h_Z(a,b)),\qquad\forall a,b\in\{0,1\}^n.
\end{equation}
Using that $(\sigma_X(a)\sigma_Z(b))^\dagger = (-1)^{a\cdot b} \sigma_X(a)\sigma_Z(b)$, the same condition must hold of the right-hand side of~\eqref{eq:conj-cliff}, thus $h_X(a,b)\cdot h_Z(a,b) = a\cdot b\mod 2$. 
 To any family of observables $\{X(a),Z(b),\,a,b\in\{0,1\}^n\}$ we associate,  for $a,b\in\{0,1\}^n$,
\begin{equation}\label{eq:def-control-c}
A(a,b) = i^{a\cdot b}X(a)Z(b), \qquad B(a,b) = i^{a\cdot b}X(h_X(a,b))Z(h_Z(a,b)),
\end{equation}
where the phase $i^{a\cdot b}$ is introduced to ensure that $A(a,b)$ and $B(a,b)$ are observables. Define $C(a,b)$ in terms of $A(a,b)$ and $B(a,b)$ as in~\eqref{eq:def-xr}. 
%Each of $A$ and $B$ can be written in a unique way as a product of commuting single-qubit Pauli observables $\{X,Y,Z\}$. 
The Clifford conjugation test aims to test for the conjugation relation $RA(a,b)R^\dagger = B(a,b)$, for all (in fact, on average over a randomly chosen) $(a,b)$. For this, we first need a test that ensures $A(a,b)$ and $B(a,b)$ themselves have the correct form, in terms of a tensor product of Pauli observables. Such a test was introduced in~\cite{natarajan2016robust}, where it is called ``Pauli braiding test''. The test certifies the Pauli relations 
\begin{align*}
& {\paulin}\{X,Y,Z\} = \Big\{ W(a)\in\Obs,\;W \in \{X,Y,Z\}^n,\,a\in\{0,1\}^n\Big\} \\
&\qquad\cup \Big\{W(a)W'(a')=(-1)^{|\{i:\,W_i\neq W'_i \wedge a_ia'_i=1\}|} W'(a')W(a),\;\forall W,W' \in \{X,Y,Z\}^n,\,\forall a,a'\in\{0,1\}^n\Big\}\\
&\qquad \cup\Big\{ W(a)W(a')=W(a+a'),\;\forall a,a'\in\{0,1\}^n\Big\}.
\end{align*}
The Pauli braiding test is recalled in Appendix~\ref{sec:pauli-group}. The original test from~\cite{natarajan2016robust} only allows to test for tensor products of $\sigma_X$ and $\sigma_Z$ Pauli observables, and we extend the test to include Pauli $\sigma_Y$. This requires us to provide a means to accommodate the phase ambiguity discussed earlier; see Appendix~\ref{sec:e-pbt} for details. 

Building on the Pauli braiding test and the conjugation test from the previous section, the Clifford conjugation test $\conjc(W)$ described in Figure~\ref{fig:conjugation-test-2} 
 provides a test for the set of relations 
\begin{align}
\conjr_{h_S,h_X,h_Z}\{R\} &= \paulin\{X,Y,Z\}  \cup \{R\in \Unitary\} \cup \{\Delta\in\Obs\}\notag\\
&\qquad \cup \big\{ R X(a)Z(b)R^\dagger = \Delta^{h_S(a,b)}X(h_X(a,b))Z(h_Z(a,b)),\,\forall a,b\in\{0,1\}^n\big\} \notag\\
&\qquad \cup \big\{ \Delta X(a) = X(a)\Delta,\,\Delta Z(b)=Z(b)\Delta,\,\forall a,b\in\{0,1\}^n\big\}.\label{eq:def-hr}
\end{align}
Note the presence of the observable $\Delta$, which arises from the conjugation ambiguity in the definition of $Y$ (see Lemma~\ref{lem:xyz-rigid}). 

\begin{figure}[H]
\rule[1ex]{16.5cm}{0.5pt}\\
Test~\conjc(R): 
\begin{itemize}
    \item Input: $R \in \mathcal{C}_n$ an $n$-qubit Clifford unitary. 	Let $h_S,h_X,h_Z$ be such that~\eqref{eq:conj-cliff} holds, and $A(a,b),B(a,b)$ the observables defined in~\eqref{eq:def-control-c}. 
    \item Relations: $\conjr_{h_S,h_X,h_Z}\{R\}$ defined in~\eqref{eq:def-hr}. 
    \item Test: execute each of the following with equal probability
\begin{enumerate}
\item[(a)] Execute test $\pbt(X,Y,Z)$ on $(n+1)$ qubits, where the last qubit is called the ``control'' qubit;
%\item[(b)] Execute test $\perm(S,n)$ with $S = \{X,Y,Z\}$ (and $k=1$);
\item[(b)] Select $a,b\in\{0,1\}^n$ uniformly at random. Let $C(a,b)$ be the observable defined from $A(a,b)$ and $B(a,b)$ in~\eqref{eq:def-xr}, with the block structure specified by the control qubit. Execute test $\conj\{A(a,b),B(a,b),R\}$. In the test, to specify query $A(a,b)$ or $B(a,b)$, use the same label as for the same query when it is used in part (a) (which exists since by~\eqref{eq:def-control-c}, each of these observables can be expressed as a string in $\{I,X,Y,Z\}^n$).
%\item[(c)] Ask one player to measure either $X_W$ (the same observable used in $\conj(W))$, or $W$, or $X(e_{2n+1})$, and the other to measure the observables $(W, X(e_{n+1}))$, where in all cases the $X$ acts on the control qubit (the $(n+1)$-st). Check that the outcomes are consistent.
\end{enumerate}
\end{itemize}
\rule[2ex]{16.5cm}{0.5pt}\vspace{-0cm}
\caption{The Clifford conjugation test, $\conjc(R)$.}
\label{fig:conjugation-test-2}
\end{figure}

Given an $n$-qubit  Clifford unitary $R$ we let $\hat{\tau}_R$ be an $n$-qubit Clifford unitary which satisfies~\eqref{eq:conj-cliff}, where for any location $i\in\{1,\ldots,n\}$ such that $a_i=b_i=1$ we replace $\sigma_X\sigma_Z$ by $\tau_Y = \sigma_Y \otimes (i\Delta_Y)$. Note that $\tau_R$ is only defined up to phase; we make an arbitrary choice of representative. As an example, in this notation we have $\tau_F = (-\sigma_X + \sigma_Y \otimes \Delta_Y)/\sqrt{2}$. We also define $\tau_{X_R}$ from $\tau_R$ as $X_R$ from $R$ in~\eqref{eq:def-xr}, with the block structure specified by the $(n+1)$-st qubit. 


\begin{lemma}\label{lem:cliff-conj}
Let $R$ be an $n$-qubit Clifford unitary and $h_S,h_X,h_Z$ such that~\eqref{eq:conj-cliff} holds. Suppose a strategy for the players succeed with probability at least $1-\eps$ in test $\conjc(R)$. Let $V_A:\mH_\reg{A} \to ((\C^2)^{\otimes (n+1)})_{\reg{A'}}  \otimes {\mH}_{\hat{\reg{A}}}$ be the isometry from Lemma~\ref{lem:xyz-rigid}.  
Then there exists a unitary $\Delta_R$ on $\hat{\mH}_{\hat{\reg{A}}}$, commuting with $\hat{\tau}_R$, such that letting  
$$ \tau_R \,=\, \hat{\tau}_{R}(\Id \otimes \Delta_R),$$
we have $R \simeq_{\poly(\eps)} \tau_R$ under the same isometry. 
\end{lemma}

Note that the observable $\Delta_R$ in the lemma takes the phase ambiguity present in the definition of $\hat{\tau}_R$ into account. 
 We stated the lemma with an approximation guarantee that depends polynomially in $\eps$. We believe it should be possible to obtain the right guarantee $O(\sqrt{\eps})$ from our proof. 

\begin{proof}[Proof sketch.]
 Completeness of the test is clear, as players making measurements on $(n+1)$ shared EPR pairs using standard Pauli observables, $R$, and $C(a,b)$ defined in~\eqref{eq:def-xr} with $A(a,b)$ and $B(a,b)$ as in~\eqref{eq:def-control-c} will pass all tests with probability $1$. 

Next we show soundness. Let $V_D$ be the isometries that follow from part (a) of the test and Lemma~\ref{lem:xyz-rigid}.
% success in part (a) of the test implies the existence of local isometries $V_A,V_B$ and observables $X(a,a')\approx_{\sqrt{\eps}} \sigma_X(a,a')$, $Y(c,c') \approx_{\sqrt{\eps}} \sigma_Y(c,c')\otimes \prod_i \Delta_i^{c_i}$ and $Z(b,b')\approx_{\sqrt{\eps}}\sigma_Z(b,b')$ for $a,b,c\in\{0,1\}^n$ and $a',b',c'\in\{0,1\}$. Using Lemma~\ref{lem:perm-test}, the permutation test from part (b) further implies that $\prod_i \Delta_i^{c_i} \approx_{\sqrt{\eps}} \Delta^{|c|}$ for some observable $\Delta$ independent of $i$. 
According to~\eqref{eq:def-control-c}, $A(a,b)$ and $B(a,b)$ can each be expressed (up to phase) as a tensor product of $X,Y,Z$ operators, where the number of occurrences of $Y$ modulo $2$ is $a\cdot b$ for $A(a,b)$ and $a\cdot b + h_S(a,b)$ for $B(a,b)$. Thus the labels used to specify the observables in $A(a,b)$ and $B(a,b)$ in part (b), together with the analysis of part (a) and Lemma~\ref{lem:xyz-rigid}, imply that up to phase, 
$$A(a,b) \simeq_{\sqrt{\eps}} \sigma_X(a)\sigma_Z(b) \otimes \Delta_Y^{a\cdot b}\qquad\text{and}\qquad B(a,b) \simeq_{\sqrt{\eps}} \sigma_X(h_X(a,b)) \sigma_Z(h_Z(a,b)) \otimes \Delta_Y^{a\cdot b + h_S(a,b)},$$
under the same isometry. 
Applying the analysis of the conjugation test given in Lemma~\ref{lem:conj} shows that $X_R$ must have the form in~\eqref{eq:def-xr}, for some $R$ that approximately conjugates $A(a,b)$ to $B(a,b)$, for all $a,b\in\{0,1\}^n$.

Let $\hat{\tau}_R$ be as defined in the paragraph preceding the lemma. Note that $\hat{\tau}_R$ acts on $\mH_\reg{A'}$ and $\mH_{\hat{\reg{A}}}$. After application of the isometry, $R$ has an expansion
\begin{equation}\label{eq:r-1}
R \simeq \sum_{a,b} \,\hat{\tau}_R \cdot\big(\sigma_X(a)\sigma_Z(b) \otimes \Delta_R(a,b)\big),
\end{equation}
 for arbitrary $\Delta_R(a,b)$ on $\mH_{\hat{\reg{A}}}$ that commute with $\Delta_Y$; such an expansion exists for any operator. 
Using the approximate version of~\eqref{eq:conj-cliff} certified by the conjugation test (Lemma~\ref{lem:conj}), for any $a,b$, 
$$ R V_A^\dagger \big(\sigma_X(a)\sigma_Z(b)\otimes \Delta_Y^{a\cdot b}\big) V_A \approx V_A^\dagger \big(\sigma_X(h_X(a,b))\sigma_Z(h_Z(a,b)) \otimes \Delta_Y^{a\cdot b + h_S(a,b)}\big)V_A R.$$
Expanding out $R$ and using the consistency relations between the two provers, 
\begin{align*}
\sum_{c,d} \hat{\tau}_R \Big(\sigma_X(c)\sigma_Z(d) &\otimes \Delta_R(c,d)\Big) \otimes \Big((-1)^{a\cdot b}\sigma_X(a)\sigma_Z(b)\otimes \Delta_Y^{a\cdot b}\Big)\\
& \approx \sum_{c,d} \Big(\sigma_X(h_X(a,b))\sigma_Z(h_Z(a,b)) \otimes \Delta_Y^{a\cdot b + h_S(a,b)}\Big) \,\hat{\tau}_R \, \Big(\sigma_X(c)\sigma_Z(d) \otimes \Delta_R(c,d)\Big) \otimes \Id.
\end{align*}
Using the conjugation relations satisfied, by definition, by $\hat{\tau}_R$,
\begin{align*}
\sum_{c,d} \Big(\sigma_X(a+c)\sigma_Z(b+d) &\otimes \Delta_R(c,d)\Big) \otimes \Big(\Id \otimes \Delta_Y^{a\cdot b}\Big)\\
& \approx \sum_{c,d} \Big((-1)^{a\cdot d  +b\cdot c} \sigma_X(a+c)\sigma_Z(b+d) \otimes \Delta_Y^{a\cdot b } \Delta_R(c,d)\Big) \otimes \Id .
\end{align*}
If $(c,d)\neq 0$ we can always find $(a,b)$ such that $a\cdot b = 0$ but $a\cdot d + b\cdot c = 1$, in which case the above, using that $\{\sigma_X(a)\sigma_Z(b) \otimes \Id \ket{\EPR}\}$ are orthogonal for different $(a,b)$, implies $\Delta_R(c,d)=-\Delta_R(c,d)$, hence $\Delta_R(c,d)\approx 0$ whenever $(c,d)\neq (0,0)$. 
Considering $(a,b)$ such that $a\cdot b=1$ implies that $\Delta_R(0,0)$ approximately commutes with $\Delta_Y$.

Finally,  the form for $X_R$ specified in the lemma follows since the control qubit used for the conjugation test is the $(n+1)$-st qubit certified by part (a) of the test.
\end{proof}

\subsection{The $n$-fold two-qubit Clifford group}
\label{sec:n-2-clifford}

We turn to testing observables in the $n$-fold direct product of the Clifford group $\cliffordgb$. Although the test can be formulated more generally, for our purposes it will be sufficient to specialize it to the case where each element in the direct product is an observable taken from the set  $\Sigma = \{X,Y,Z,F,G\}$.

\begin{figure}[H]
\rule[1ex]{16.5cm}{0.5pt}\\
Test~$\cliff(\Sigma,n)$:
\begin{itemize}
    \item Input: An integer $n$ and a subset $\Sigma = \{X,Y,Z,F,G\}$ of the single-qubit Clifford group. 
    \item Test: Select $W \in \Sigma^n$ uniformly at random. Execute each of the following with equal probability:
\begin{enumerate}
%\item[(a)] Execute test $\pbt(X,Y,Z)$ on $2n+1$ qubits;
\item[(a)] Execute the test $\conjc(W)$;
\item[(b)] Send one player either the query $W$, or $X_W$ and the other $(W,X(e_{n+1}))$, where $e_{n+1}$ indicates the control qubit used in part (a). Receive one bit from the first player, and two from the second. Check that the first player's answer is consistent with the second player's first answer bit in case the query was $W$, and with the second in case the query was $X_W$. 
\item[(c)] Let $S$ and $T$ be subsets of the positions in which $W_i=F$ and $W_i=G$ respectively, chosen uniformly at random. Let $R = Y(\sum_{i\in S\cup T} e_i)$. Execute test $\conj(W,W', R)$.
\item[(d)] Execute test $\pbt(F,G,Z)$ on $n$ qubits. 
\item[(e)] Let $S$ and $T$ be subsets of (non-overlapping) pairs of positions in which $W_i=F$ and $W_i=G$ respectively, chosen uniformly at random. Send one player the query $W$, with entries in $(i,j) \in S\cup T$ removed and replaced by $\SWAP_{i,j}$. With probability $1/2$, send the other player the query $W$, and check the natural consistency relations. With probability $1/2$, execute an independent copy of the Bell measurement test $\bellt$ (Figure~\ref{fig:bell}) between the first and second players, in each pair of qubits in $S\cup T$.  
\end{enumerate}
\end{itemize}
\rule[2ex]{16.5cm}{0.5pt}
\caption{The $n$-qubit Clifford test, $\cliff(\Sigma,n)$.}
\label{fig:clifford-test-3}
\end{figure}

The test is described in Figure~\ref{fig:clifford-test-3}. It is divided in five parts. Part (a) of the test executes  $\conjc(W)$ to verify that an observable $W\in\Sigma^n$ satisfies the appropriate Pauli conjugation relations~\eqref{eq:conj-cliff}. Note that a priori test $\conjc(W)$ only tests for the observable $X_W$ obtained from $W$ in blocks as $X_R$ from $R$ in~\eqref{eq:def-xr} (indeed, in that test $W$ need not be an observable). Thus part (b) of the test is introduced to verify that $X_W \approx W X(e_{n+1})$, where the $(n+1)$-st qubit is the one used to specify the block decomposition relating $X_W$ to $W$.  The result of parts (a) and (b) is that, under the same isometry as used to specify the Pauli $X$ and $Z$, $W\simeq \hat{\tau}_W \cdot (\Id\otimes \Delta_W)$, according to the same decomposition as shown in Lemma~\ref{lem:cliff-conj}. The goal of the remaining three parts of the test is to verify that $\Delta_W = \Delta_F^{|\{i: W_i \in \{F,G\}\}|}$, for a single observable $\Delta_F$. For this, part (c) of the test verifies that $\Delta_W$ only depends on the locations at which $W_i\in\{F,G\}$, but not the specific values. Part (d) verifies that $\Delta_W \approx \prod_{i: W_i\in \{f,G\}} \Delta_i $ for commuting observables $\Delta_i$. Finally, part (e) checks that $\Delta_i$ is (approximately) independent of $i$. 

\begin{theorem}\label{thm:clifford-ntest}
Suppose a strategy for the players succeeds in test $\cliff(\Sigma,n)$ (Figure~\ref{fig:clifford-test-3}) with probability at least $1-\eps$. Then  for $D\in\{A,B\}$ there exists an isometry 
$$V_D: \mathcal{H}_\reg{D} \to (\C^2)^{\otimes (n+1)}_{\reg{D}'} \otimes \hat{\mH}_{\hat{\reg{D}}}$$
such that 
\begin{equation}\label{eq:psi-epr}
\big\| (V_A \otimes V_B) \ket{\psi}_{\reg{AB}} - \ket{\EPR}_{\reg{A}'\reg{B}'}^{\otimes (n+1)} \ket{\aux}_{\hat{\reg{A}}\hat{\reg{B}}} \big\|^2 = O(\sqrt{\eps}),
\end{equation}
and %for any $W \in \mathcal{G}$ there is a $\Delta_W \in \Obs(\hat{\mH})$ such that the $\Delta_W$ pairwise commute and 
\begin{equation}\label{eq:clifford-ntest-close}
\Es{W\in\Sigma^n,\,c\in\{0,1\}^n} \big\| \Id_A \otimes \big( V_B W(c) - \tau_{W}(c) V_B\big)   \ket{\psi}_{\reg{AB}} \big\|^2 = O(\poly(\eps)).
\end{equation}
Here $\tau_W$ is defined from $W$ as in Lemma~\ref{lem:cliff-conj}, with $\Delta_{W_i} = \Id$ if $W_i\in \{X,Y,Z\}$ and $\Delta_{W_i} = \Delta_F$ if $W_i\in\{F,G\}$, where $\Delta_F$ is an  observable on ${\mH}_{\hat{\reg{B}}}$ that commutes with $\Delta_Y$. 
\end{theorem}

\begin{proof}[Proof sketch]
The existence of the isometry, as well as~\eqref{eq:psi-epr} and~\eqref{eq:clifford-ntest-close} for $W \in \{I,X,Y,Z\}^{n}$, follows from the test $\pbt(X,Y,Z)$, executed as part of the Clifford conjugation test from part (a), and Lemma~\ref{lem:xyz-rigid}. 
Using part (a) of the test and Lemma~\ref{lem:cliff-conj} it follows that every $W \in \Sigma^n$ is mapped under the same isometry to $W \simeq_{\sqrt{\eps}} \tau_W = \hat{\tau}_W \otimes \Delta_W$, where $\hat{\tau}_W$ is as defined in the paragraph preceding the lemma  and  $\Delta_W$ is an observable on $\hat{\reg{A}}$ which may depend on the whole string $W$;  here we also use the consistency check in part (b) to relate  the observable $X_W$ used in the Clifford conjugation test with the observable $W$ used in part (c). Note that from the definition we can write $\hat{\tau}_W = \otimes_i \hat{\tau}_{W_i}$, where in particular $\hat{\tau}_X = \sigma_X$, $\hat{\tau}_Z = \sigma_Z$ and $\hat{\tau}_Y = \sigma_Y \otimes \Delta_Y$.

The analysis of the conjugation test given in Lemma~\ref{lem:conj} shows that success with probability $1-O(\eps)$ in part (c) of the test implies the relations 
\begin{align*}
 \hat{\tau}_{W'} \otimes \Delta_W &= \tau_R \hat{\tau}_W \tau_R^\dagger \otimes \Delta_W \\
&  \approx_{\sqrt{\eps}} \hat{\tau}_{W'} \otimes \Delta_{W'},
\end{align*}
where the first equality is by definition of $R$, and uses that $\tau_Y = \sigma_Y \otimes \Delta_Y$ and $\Delta_Y$ commutes with $\Delta_W$, and the approximation holds as a consequence of the conjugation test and should be understood on average over a uniformly random choice of $W\in \Sigma^n$. Thus $\Delta_W$ depends only on the locations at which $W_i \in \{F,G\}$, but not on the particular values. 

Part (d) of the test is played only so that we obtain Pauli linearity relations which allow us to decompose $\Delta_W = \prod_{i: W_i\in\{F,G\}} \Delta_i$ for commuting observables $\Delta_i$. Finally, part (e) of the test enforces that on average over a random pair $(i,j)$ it holds that $\Delta_i\Delta_j \approx \Id$. This implies that $\Delta = \Delta_F$ can be made independent of $i$, and concludes the proof. (We refer to the proof of Lemma~\ref{lem:xyz-rigid}, which considers similar estimates, for more details.)
\end{proof}


\subsection{Post-measurement states}

We give a first corollary of Theorem~\ref{thm:clifford-ntest} which expresses its conclusion~\eqref{eq:clifford-ntest-close} in terms of the post-measurement state of the first player. This corollary will be used in the analysis of the leash protocol from Section~\ref{sec:leash}. It will be convenient (though not necessary) to assume that all observables in $\mathcal{G}$ come in non-trivial commuting pairs $(W,W')$, i.e. each of $W, W'$ and $WW'=W'W$ has rank $2$. This enforces that the two observables specify an orthonormal basis of $\C^2\otimes \C^2$. We write $\{\sigma_{W,W',+1}^{u,u'}\}_{(u,u')\in\{\pm 1\}}$ for the associated family of rank-one projections, and $\{\sigma_{W,W',-1}^{u,u'}\}_{(u,u')\in\{\pm 1\}}$ for the rank-one projections obtained from the pair $(\overline{W},\overline{W'})$. 

\begin{corollary}\label{cor:clifford-rigid}
Let $\eps>0$, $n$ an integer and $\mathcal{G} \subseteq \cliffordgb\times\cliffordgb$ a finite set of pairs of commuting Clifford observables. Suppose a strategy for the players succeeds with probability $1-\eps$ in test $\cliff(\mathcal{G},n)$. Then for $D\in\{A,B\}$ there exists an isometry 
$$V_D: \mathcal{H}_\reg{D} \to (\C^2)^{\otimes 2n} \otimes {\mH}_{\hat{\reg{D}}}$$
such that
$$ \big\| (V_A \otimes V_B) \ket{\psi}_{\reg{AB}}  - \ket{\EPR}^{\otimes 2n} \otimes \ket{\aux}_{\hat{\reg{A}}\hat{\reg{B}}} \big\|^2 = O(\sqrt{\eps}),$$
and orthogonal density matrices $\tau_\epsilon$ on $\hat{\reg{A}}$, for $\epsilon\in\{-1,1\}$, such that
$$ \mathop{\textsc{E}}_{W\in\mathcal{G}^n} \sum_{u\in\{\pm 1\}^{2n}} \Big\| V_A \Tr_{\reg{B}}\big((\Id_A \otimes W_{\reg{B}}^u) \proj{\psi}_{\reg{AB}} (\Id_A \otimes W_{\reg{B}}^u)^\dagger\big) V_A^\dagger - \sum_{\epsilon\in\{-1,1\}} \Big( \bigotimes_{i=1}^n \frac{\sigma_{W_{2i},W_{2i+1},\epsilon}^{u_{2i},u_{2i+1}}}{4}\Big)\otimes \frac{\tau_\epsilon}{2}   \Big\|_1 = O(\eps^c),$$
for some universal constant $c>0$. 
\end{corollary}

\begin{proof}
From Theorem~\ref{thm:clifford-ntest} we get isometries $V_A$, $V_B$ and an observable $\Delta_Y$ on ${\mH}_{\hat{\reg{A}}}$ such that the conclusions of the theorem hold. Write the eigendecomposition $\Delta_Y = \Delta_Y^{+1}-\Delta_Y^{-1}$ and $\Delta_Y = \Delta_H^{+1}-\Delta_H^{-1}$, and for $\lambda \in \{\pm 1\}^2$ let
$$\tau_{\lambda} = \Tr_{\hat{\reg{B}}}\big( \big(\Id_{\hat{\reg{A}}}\otimes \Delta_Y^{\lambda_1}\Delta_H^{\lambda_2} \big)\proj{\aux}\big(\Id_{\hat{\reg{A}}}\otimes \Delta_Y^{\lambda_1}\Delta_H^{\lambda_2}\big)\big).$$
Using that $\Delta_Y$ and $\Delta_H$ commute and satisfy 
$$\Delta_Y \otimes \Delta_Y \ket{\aux} \approx \Delta_H\otimes \Delta_H \ket{aux} \approx \ket{\aux}$$
 it follows that the (sub-normalized) densities $\tau_{\lambda}$ have (approximately) orthogonal support. 
For $G=(G_{i,1},G_{i,2})_{i=1,\ldots,n}\in \mathcal{G}^{\otimes n}$ and $(b,c)\in \{0,1\}^n \times \{0,1\}^n$ the observable $G(b,c) = \otimes_i (G_{i,1}^{b_i}G_{i,2}^{c_i})$ can be expanded in terms of a $4n$-outcome projective measurement $\{G^{u,v}\}$ as 
$$G(b,c) = \sum_{(u,v)\in \{0,1\}^n\times\{0,1\}^{n}}  (-1)^{u\cdot b+ v\cdot c} \,G^{u,v}.$$
Similarly, by definition the ``honest'' projective measurement associated with the Pauli observables $\tau_G(b,c) = \otimes_i (\tau_{G_{i,1}}^{b_i}\tau_{G_{i,2}}^{c_i})$ are 
$$\tau_G^{u,v} = \bigotimes_i \,\Big(\Es{(b,c)\in \{0,1\}^n\times\{0,1\}^n} \,(-1)^{u\cdot b + v\cdot c} \tau_G(b,c)\Big).$$ 
Thus,
\begin{align}
\Es{(b,c)\in\{0,1\}^n\times\{0,1\}^n} & \big\| \Id_A \otimes \big(  G(b,c) - V_B^\dagger \tau_{G}(b,c) V_B\big)   \ket{\psi}_{\reg{AB}} \big\|^2\notag\\
&= \Es{(b,c)\in\{0,1\}^n\times\{0,1\}^n}\Big\| \sum_{u,v} (-1)^{u\cdot b+v\cdot c} \Id_A \otimes \big(  G^{u,v} - V_B^\dagger\tau_{G}^{u,v} V_B\big)   \ket{\psi}_{\reg{AB}} \Big\|^2    \notag\\
&= \sum_{(u,v)\in\{0,1\}^n\times\{0,1\}^n}\big\|  \Id_A \otimes \big(  G^{u,v} - V_B^\dagger\tau_{G}^{u,v} V_B\big)   \ket{\psi}_{\reg{AB}} \big\|^2, \label{eq:ntest-close-b}  
\end{align}
where the second line is obtained by expanding the square and using $\Es{(b,c)\in\{0,1\}^n\times\{0,1\}^n} (-1)^{u \cdot b+v\cdot c} = 1$ if $(u,v)=(0^n,0^n)$, and $0$ otherwise. Using~\eqref{eq:clifford-ntest-close}, the expression in~\eqref{eq:ntest-close-b}, when averaged over all $G\in\mathcal{G}^{\otimes n}$, is bounded by $O(\eps^c)$ for some constant $c$. Using the Fuchs-van de Graaf inequality and the fact that trace distance cannot increase under tracing out, we obtain 
$$\Es{G\in\mathcal{G}^{\otimes n}}\sum_{u,v} \Big\| \Tr_\reg{B}\big( (\Id_\reg{A} \otimes G^{u,v})\proj{\psi}(\Id_\reg{A} \otimes G^{u,v})^\dagger \big) - \Tr_\reg{B}\big( (\Id_\reg{A} \otimes \tau_G^{u,v})\proj{\psi}(\Id_\reg{A} \otimes \tau_G^{u,v})^\dagger \big) \Big\|_1 = O(\eps^{c/2}).$$
To conclude the bound claimed in the theorem it only remains to use~\eqref{eq:psi-epr}, with the factor $\frac{1}{4}$ coming from the reduced density matrix of two EPR pairs.
\end{proof}


\subsection{Tomography}
\label{subsec:tomography}

Theorem~\ref{thm:clifford-ntest} and Corollary~\ref{cor:clifford-rigid} show that success in test $\cliff(\mathcal{G},n)$ gives us control on the players' observables and post-measurement states in the test. This allows us to use one of the players to perform some kind of limited tomography (limited to post-measurement states obtained from measurements in $\mathcal{G}$), that will be useful for our analysis of the dog-walker protocol from Section~\ref{sec:dog-walker}.

 Consider the test $\tom(\mathcal{G},n)$ described in Figure~\ref{fig:tomography-test}. In this test, one player is sent a random question $W\in\mathcal{G}^n$ distributed as in the test $\cliff(\mathcal{G},n)$ (i.e. uniformly at random), and asked to report his outcomes $a\in\Lambda$. From  Corollary~\ref{cor:clifford-rigid} it follows that the second player's post-measurement state is close to a state consistent with the first player's reported outcomes. Now suppose the second player, who is not sent any information, is allowed to report an arbitrary string $W'\in \mathcal{G}^n$, together with outcomes $u\in\Lambda$. Suppose also that for each $i$, $u_i=a_i$ whenever $W'_i=W_i$. Since the latter condition is satisfied by a constant fraction of $i\in\{1,\ldots,n\}$, irrespective of $W'$, with very high probability, it follows that the only possibility for the second player to satisfy the condition is to actually measure his qubits precisely in the basis that he indicates. This allows us to check that a player performs the measurement that he claims, even if the player has the choice of which measurement to report. 

\begin{figure}[H]
\rule[1ex]{16.5cm}{0.5pt}\\
Test~$\tom(\mathcal{G},n)$: $\mathcal{G} \subseteq \cliffordgb \cap \Obs(\C^2\otimes \C^2)$; $n$ an even integer.  \\
The verifier performs the following one-round interaction with two
players.  With equal probability,
\begin{enumerate}
%\item[(a)] Execute test $\pbt(X,Y,Z)$ on $2n+1$ qubits;
\item[(a)] Execute the test $\cliff(\mathcal{G},n)$; 
\item[(b)] Select $W\in\mathcal{G}^n$ uniformly at random. Send $W$ to the first player, and the signal ``$\tom(\mathcal{G},n)$'' to the second. Receive $a$ from the first player, and $W'\in\mathcal{G}^n$ and $u$ from the second. Accept only if $a_i=u_i$ whenever $W_i=W'_i$. 
\end{enumerate}
\rule[2ex]{16.5cm}{0.5pt}\vspace{-1cm}
\caption{The $2n$-qubit tomography test, $\tom(\mathcal{G},n)$.}
\label{fig:tomography-test}
\end{figure} 

\begin{corollary}\label{cor:clifford-rigid-adaptive}
Let $\eps>0$, $n$ an even integer and $\mathcal{G} \subseteq (\cliffordgb \cap \Obs(\C^2 \otimes \C^2))\cup\{B,C\}$. Let $\hat{\mathcal{G}} = (\mathcal{G}\backslash\{B,C\}) \cup \{XX,ZZ,YY\}$. \anote{I think $B,C$ are not defined in the main text.}Suppose a strategy for the players succeeds with probability $1-\eps$ in test $\tom(\hat{\mathcal{G}},n)$. Let $V_A,V_B$ be the isometries specified in Corollary~\ref{cor:clifford-rigid}. Let $\{Q^{W',u}\}$ be the projective measurement applied by the second player in part (b) of the test. Then there exists a distribution $q$ on $\hat{\mathcal{G}}^n \times \{\pm 1\}$ such that 
\begin{align*}
 \sum_{W'\in\hat{\mathcal{G}}^n} \sum_{u\in\prod_i \Lambda_i} &\Big\| \Tr_{\reg{A}\hat{\reg{B}}} \big((\Id_A \otimes V_B Q^{W',u} )\proj{\psi}_{\reg{AB}} (\Id_A \otimes V_B Q^{W',u})^\dagger\big)\\
&\hskip3cm - \sum_{\epsilon\in\{-1,1\}}  q(W',\epsilon)  \Big( \bigotimes_{i=1}^n \frac{1}{4}\sigma_{W'_i,\epsilon}^{u_i}\Big) \Big\|_1 = O(\sqrt{\eps}),
\end{align*}
where the notation is the same as in Corollary~\ref{cor:clifford-rigid}. 
\end{corollary}

\begin{proof}
Success in part (a) of the test allows us to apply Corollary~\ref{cor:clifford-rigid-adaptive}. For any $(W',u)$ let $\rho_{\reg{A'}}^{W',u}$ be the post-measurement state on the first player's space, conditioned on the second player's answer  in part (b) of the test being $(W',u)$, and after application of the isometry $V_A$. Using~\eqref{eq:psi-epr} and the properties of the maximally entangled state,  this is approximately the same as the second player's post-measurement state, 
\begin{equation}\label{eq:tom-1}
\big\|\rho_{\reg{A'}}^{W',u} - \Tr_{\reg{A}\hat{\reg{B}}}\big((\Id_A\otimes V_B Q^{W',u} )\proj{\psi}_{\reg{AB}}(\Id_A\otimes V_B Q^{W',u} )^\dagger\big)\big\|_1 \,=\, O(\sqrt{\eps}).
\end{equation}
On the other hand, using that for any $i\in\{1,\ldots,n\}$, $W_i=W'_i$ with constant probability $|\hat{\mathcal{G}}|^{-1}$, 
it follows from~\eqref{eq:clifford-ntest-close} in Theorem~\ref{thm:clifford-ntest} that success in part (b) of the test implies the condition
\begin{equation}\label{eq:tom-2}
\sum_{W',\epsilon }\Tr(\tau_\epsilon) \sum_u\, \Tr\Big(\Big( \frac{|\hat{\mathcal{G}}|-1}{|\hat{\mathcal{G}}|}\Id + \frac{1}{|\hat{\mathcal{G}}|}\sigma_{W',\epsilon}^u \Big) \rho_{\reg{A'}}^{W',u}\Big)  = 1- O(\sqrt{\eps}). 
\end{equation}
Combining~\eqref{eq:tom-1} and~\eqref{eq:tom-2} concludes the proof, for some distribution $q(W',\epsilon) \approx \sum_u \Tr(\rho_{\reg{A'}}^{W',u})\Tr(\tau_\epsilon)$ (the approximation is due to the fact that the latter expression only specifies a distribution up to error $O(\sqrt{\eps})$.
\anote{I found the last part of this proof a bit hard to follow.}
\end{proof}


